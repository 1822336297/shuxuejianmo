\documentclass{whutmod}
\usepackage{metalogo}
\usepackage{float}
\usepackage{subfigure} 
\usepackage{url}
\usepackage[style=caspervector,backend=biber,utf8]{biblatex}
\addbibresource{wenxian.bib}
\team{23}	% 组号
\membera{刘子川}
\joba{编程}
\memberb{程宇}
\jobb{建模}
\memberc{陈荣兴}
\jobc{建模}
\hypersetup{
	colorlinks=true,
	linkcolor=black
}
           
\title{论文《高温作业专用服装设计》读后感}
%\tihao{3} % 题号

\begin{document}
	
	\maketitle
\Large   
我组学习的论文题目为《高温作业专用服装设计》,这是一篇2018年全国大学生数学建模国赛A题优秀论文。

如何根据环境条件设计相应的服装是专用服装设计面临的主要问题,该篇论文通过建立一维复合介质热传导方程,对高温作业专用服中各分层间的传热过程进行模拟分析,确定不同环境条件下作业服中的温度分布。该论文数学模型合理、具有详细的计算推导过程、图文详实、文献综述丰富而规范。其中合理的模型假设简化了服装热传导的过程,提高了模型求解的精确性。接下来对文章的各方面谈谈自己的理解和感受:

\section{文章大揽}
该篇论文在摘要部分的优点主要有三个:一是段落中的重点使用了\textbf{黑体}标识,将使用的模型方法和系数数值加粗描黑,加深了读者对模型所用方法和对应求解结果的印象;二是每个问题联系十分紧密,对问题的描述思路清晰,在问题综述、预备工作、模型建立、问题结果等各个环节都有承上启下的论证;三是最后一段对整体的模型建立,通过后续系数的灵敏性分析,点出了模型的合理性。

在文章主体部分,对模型确定的三维等方向均匀介质的热传导方程,结合题目所给实际,在问题分析中画出了模型的示意图,将模型图像化,使文字的表述更加形象化,加深读者的理解。再结合其初始条件和边界条件,对原有的模型方程进一步改进及求解。模型求解中的系数推导过程也是一大亮点,层层推进,承上启下,科学严谨。

在模型的评价部分,该论文对模型的各分层厚度做出灵敏性分析。利用图表的方式向读者展示了模型的合理性,并说明对热交换系数$h1$的不敏感的原因,充分显示出模型的鲁棒性。对模型进行灵敏度分析和综合评价是我组尚欠缺的部分,今后需要加以改进。


\section{模型方法总结}
该篇论文先对原有模型三维等方向均匀介质的热传导方程进行简化得到一维复合介质热传导方程。采用$Crank-Nicholson$方法,求解出条件中未知的参数,进而根据不同的边界条件进行问题求解。该文整体对一维复合介质热传导方程模型根据不同问题的要求进行多角度的参数约束限定,并模型进行合理的简化,使得求解结果符合要求。

针对问题一,从作业专用服装设计中温度的传导出发,基于一般性热传导方程,把各分层的作业服简化成相互接触的平行无限大的的平板,建立对应一维复合介质热传导方程,采用$Crank-Nicholson$方法进行数据求解并和实验数据拟合,求解出对流热系数,再代回到参数方程中,对作业服的热传递模型求解,得到温度分布图。

针对问题二,根据经济性,把Ⅱ层的最优厚度理解为满足约束条件的最小厚度以节约成本。把问题二转变成非线性规划问题,结合一维复合介质热传导模型,基于附件数据中的Ⅱ层厚度范围进行定长$2mm$的步长搜索,在不同厚度取值下求解模型得到假人皮肤外侧温度时间变化,在目标函数上确定约束条件,得到最大温度和温度超过$44℃$对应Ⅱ层厚度关系的变化图象,最终确定合适步长搜下Ⅱ层的最优厚度。

针对问题三,文中结合了问题二的优化问题,同时考虑第Ⅱ层与Ⅳ层最优厚度,由于第Ⅳ层为空气层,先使第Ⅱ层厚度最小化,在此前提上增加约束条件来最小化第Ⅳ层的厚度。利用区域搜索算法确定边界条件下Ⅱ和Ⅳ层的厚度范围,再循环遍历找到满足条件下的点集范围。得到Ⅱ和Ⅳ层的最优厚度


\section{模型的分析}
本文整体模型从一般性热传导方程出发,将作业服各层视为相互接触的平行无限大平板,建立一维复合介质热传导方程模型,运用了偏微分方程的求解知识,使用有限差分方法数值计算,解题过程更简单化,通俗易懂。以及在给定条件下确定不同层作业服的温度分布以及最优厚度。求解过程严密、紧凑、步步有依据,模型求解的结果更科学。结合各种相互关联的多变量约束条件下下建立的最优化模型,合理的进行简化理解,解决了目标函数的最优问题。模型求解简单,具有很强的直观性,可操作性强且适用范围广。最终模型对对流换热系数的灵敏性检验验证了模型求解结果的准确性。但各分层介质的具体形状对于热传导的效率和系数的大小具有一定的影响,未充分考虑到热量传递过程中的热辐射,会增大模型在高温条件下的误差,具有一定的优化空间。以及忽略了热传递过程中的纵向传递,与实际情况下的温度传导况存在一定偏差。


\section{结语}
整篇论文的结构合理,层次分明,模型推导过程中采用了递进式的分析结构,科学严谨,逻辑性强,表达清晰,是一篇很有价值的文章。读过该论文后,我们不仅了解了热传导方程的推导求解,而且也学习到了模型建立中科学严谨的求解过程,以及合理的模型假设对于问题求解的极大帮助,这都值得我们进一步深入探讨与学习。
\end{document}