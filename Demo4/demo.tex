\documentclass{whutmod}
\usepackage{metalogo}
\usepackage{float}
\usepackage{subfigure} 
\usepackage{url}
\usepackage[style=caspervector,backend=biber,utf8]{biblatex}
\addbibresource{wenxian.bib}
\team{23}	% 组号
\membera{刘子川}
\joba{编程}
\memberb{程宇}
\jobb{建模}
\memberc{陈荣兴}
\jobc{建模}
\hypersetup{
	colorlinks=true,
	linkcolor=black
}

\title{第四次论文格式}
\tihao{4} % 题号

\begin{document}
%	这里的封面你们有问题,先注释着
%	\maketitle
	
	\begin{abstract}
“拍照赚钱”APP 是基于移动互联网的自助式劳务众包平台,使得企业可利 用大众力量,低成本、高效率地完成各种商品检查与信息搜集的任务。本文通过 建立数学模型,就 APP 中的任务定价问题进行分析,给出最优的任务定价方案。
   ~\\%~\\为换行
   
针对问题一, 针对问题的这一段总结。利用什么软件+建立什么模型+算法方法+求解的特点+主要的结果+评价及其推广。  注意黑体字\text{描黑重点关键词}。
~\\

\textbf{例子:}针对问题二,对项目的任务定价规律进行\textbf{定性与定量研究}。利用 Matlab 的 cftool 工具箱绘制出任务的经纬度坐标与定价数据的\textbf{三维拟合图},观察到任务 分布密集的地区任务定价较低。对任务的位置数据进行空间离散化处理和 K-Means 分析,将任务分布的区域等划分为若干网格区域,定义影响任务定价的 四个因子,即网格内任务数量、会员人数、会员平均完成能力、任务与中心点的 距离。运用灰色关联矩阵定量分析四个影响因子与定价的相关度,分别为 0.9710,0.9671,0.9633,0.9390。得出所定义的指标对定价相关性很高,能较好 描述定价规律。最后通过比较未完成任务与已完成任务的相关度矩阵得出距离对 任务的完成的影响是最显著的
   ~\\

针对问题三,
   ~\\
   
针对问题四,
   ~\\

本文中所提到的模型优点主要有两点:一、在与污染源头距离较短时预测抗噪能力较强;二、利用更高定位精度和鲁棒性的直线解析法,溯源追踪能力较强。
  
		\keywords{对流扩散方程\quad  直线解析法\quad  溯源算法\quad 拉普拉斯变换\quad }
		
	\end{abstract}
	
	%目录
	\tableofcontents
	\newpage	%换页符
	
	\section{问题重述}
	\subsection{问题背景}
xxxxxxxxxxxxxxxxxxxxxxxxxxxxxxxxxxxxxxxxxxxxxxxxxxxxxxx
	
	\subsection{问题提出}
	
	围绕xxxxxxx,依次提出以下问题:
	
	\begin{itemize}
		\item [(1)]
		 抄
		\item [(2)] 
		\item [(3)] 
		\item [(4)] 
	\end{itemize}
	
	\section{模型假设}
	\begin{itemize}
		\item [(1)] 就写假设就行了吧
		\item [(2)] 
		\item [(3)]
		\item [(4)]
	\end{itemize}
	
	
	\section{符号说明}
%	每行都有线的表
%	\begin{center}
%		\begin{tabular}{cc}
%			\hline
%			\makebox[0.3\textwidth][c]{符号}	&  \makebox[0.4\textwidth][c]{意义} \\ \hline
%			$C_{0}$	    &  污染源初始浓度 \\ \hline
%			$C(x,t)$	    &  污染浓度随时空变化 \\ \hline
%			$u_{x}$	    &  江河平均纵向流速 \\ \hline
%			$E_{x}$  &  铊在江河纵向弥散系数\\ \hline
%		$p$   &  面污染物纵向距离\\ \hline
%			$K_{c}$	    & 污染物降解系数  \\ \hline
%		    $a$	& 污染超标系数 \\ \hline
%		     $x$	& 距污染源的一维距离 \\ \hline
%		      $t$	& 距污染发生后的时间 \\ \hline
%		       $V_{A}$	& 溶液摩尔体积 \\ \hline
%		      $M_{B}$	& 江水的摩尔质量 \\ \hline
%		     $\mu_{B}$	& 溶剂的粘度 \\ \hline		      
%		\end{tabular}
%	\end{center}

%三线表
	\begin{table}[H]
	\label{biao} \centering
		\begin{tabular}{cccc}
			\toprule[1.5pt]
			%%表示第一列占4cm 第二列占4cm 第三列占4cm 的距离 并且这几个字都是居中对齐
			\multicolumn{1}{m{4cm}}{\centering 符号} & \multicolumn{1}{m{4cm}}{\centering 说明} & \multicolumn{1}{m{4cm}}{\centering 单位}\\
			\midrule[1pt]
			$C_{0}$	 &  污染源初始浓度 & 单位\\ 
			$C(x,t)$ &  污染浓度随时空变化 & 单位\\ 
			$u_{x}$	 &  江河平均纵向流速 & 单位\\ 
			$E_{x}$  &  铊在江河纵向弥散系数& 单位\\ 
			\bottomrule[1.5pt]
		\end{tabular}
	\end{table}
	注:未列出符号及重复的符号以出现出为准

	
	\section{一级标题}
	\subsection{二级标题}
	\subsubsection{三级标题}
	\paragraph{段落}
	\section{问题一模型的建立与求解}
	\subsection{问题的描述与分析}
	针对问题一,本题要求对武汉的人才吸引力做出量化评价,本文采用因子分析法建立模型,将通过不同的因子对变量的影响程度进行分析,建立客观的评价模型。首先根据题目要求和以往人才吸引要素研究,寻找合适的人才吸引力因素,确立要素框架并寻找各个因素相对应的数据。在将所得数据进行标准化与归一化处理后,进行$KMO$检验和巴特利特检验以验证所选变量是否适合做因子分析。通过主成分分析法确定公因子数目,求解旋转后因子荷载矩阵并对公因子命名。
	\begin{figure}[H]
		\centering
		\includegraphics[width=.8\textwidth]{figures/lct.png}
		\caption{算法流程图}\label{lct}
	\end{figure}
	\subsection{模型的建立与求解}
	\subsubsection{评价指标建立}
	\subsubsection{公因子的确定}
	标准化后数据如表一所示(附表一),表一相关系数矩阵$R$如下(附加相关系数矩阵)
	设$\lambda_{1} \geqslant \lambda_{2} \geqslant \cdots \geqslant \lambda_{p}$为样本相关系数矩阵$R$的特征值,$??$为相应的标准正交化特征向量。设$m<p$,则因子载荷矩阵$\Lambda$为
	\begin{gather}
	\Lambda=\left[\sqrt{\lambda_{1}} \eta_{1}, \sqrt{\lambda_{2}} \eta_{2}, \cdots, \sqrt{\lambda_{m}} \eta_{m}\right]
	\end{gather}
    用$\boldsymbol{R}-\boldsymbol{\Lambda} \boldsymbol{\Lambda}^{\mathrm{T}}$对角元来估计特殊因子的的方差
	\begin{gather}
	\sigma_{i}^{2}=1-\sum_{j=1}^{m} \alpha_{i j}^{2}
	\end{gather}
	得总方差解释如表三(附表三)
	由表$3$特征根知,因子$1$的特征值$λ1=$,占方差的 $\%$。由图$1$知,当提取$1、2$个公因子时,特征值变化非常明显,当提取第$n+1$个以后的公因子时,特征值变化比较小,基本趋于平缓。由此说明,提取$n$个公因子对原变量信息的刻画有显著作用。因此,在这里我们提取$n$个公共因子,这$n$个公因子的累计方差达到$\%$,即这$n$个公因子可以反映原来$?$个指标的$???\%$的信息量,可见采用前$n$个公因子对这$?$个城市吸引力进行评价是比较合适的。
	\subsubsection{未转轴的因子载荷矩阵}
	表 4 中的每个数据表示了相应因子变量对相应原变量的相对重要程度。由于得到的公共因子与各指标的载荷分布归类比较困难,需要对因子载荷矩阵进行正交旋转,这里运用方差最大正交旋转法,得到旋转后的因子载荷矩阵表(4)。
	表$3$是初始因子载荷矩阵,由此可写出因子分析模型的如下:
	\begin{gather}
	\sigma_{i}^{2}=1-\sum_{j=1}^{m} \alpha_{i j}^{2}%(附线性方程公式)
	\end{gather}
	\subsubsection{未转轴的因子载荷矩阵}
	通过凯撒正态化最大方差法,旋转在$?$次迭代后已收敛。得表$5$(附表五)
	根据表 5 发现,旋转后的因子系数已经明显向两极分化,有了更鲜明的实际意义。因子载荷的绝对值越大,则表明该因子与变量的重叠性越高,在解释因子的时候就越重要。
	(因子命名)
	\subsubsection{求得因子得分和综合绩效得分}
	采用回归法估计因子得分系数,并输出因子得分系数矩阵。具体结果见表$6$
	由估计出的因子得分,可以量化描述城市人才吸引力水平,利用因子得分可以从不同角度对城市人才吸引力水平进行比较分析。为了便于对各城市进行人才吸引力评价,现利用各城市的因子得分表计算综合得分,吸引力水平的获取是基于总方差分解表中旋转后各因子的方差贡献率及计算所得的城市各因子得分获取的。其计算公式如下:
	\begin{gather}
	\sigma_{i}^{2}=1-\sum_{j=1}^{m} \alpha_{i j}^{2}%(附结果计算公式)
	\end{gather}
	计算结果见表$7$
	\subsection{结果分析}


%	\parencite{geng2019novel}为文献引用
	查阅资料可得\parencite{geng2019novel}
	
%	~\ref{llllll}~为图表引用,中间为为图表\label{}
具体浓度分布如图~\ref{llllll}~所示:

%	图片插入,大小调整[width=.5\textwidth]	[width=.7\textwidth]等
	\begin{figure}[H]
		\centering
		\includegraphics[width=\textwidth]{figures/matlab.png}
		\caption{浓度分布随时间推移过程图}\label{llllll}
	\end{figure}

	\section{问题二模型的建立与求解}
	\subsection{问题的描述与分析}
	问题分析写\textbf{流程图!!}其流程图如图~\ref{lct}~所示:
			\begin{figure}[H]
	\centering
	\includegraphics[width=.8\textwidth]{figures/lct.png}
	\caption{算法流程图}\label{lct}
	\end{figure}
	\subsection{模型的建立与求解}
	

	\section{问题三模型的建立与求解}
	
	
	\subsection{问题的描述与分析}

	\subsection{模型的建立与求解}

	\section{问题四模型的建立与求解}
	
	\subsection{问题的描述与分析}
	
	\subsection{模型的建立与求解}



	\section{模型的评价}
	\subsection{模型的优点}
xxxxxxxxxxxxxxxxxxxxxxxxx
	
	\subsection{模型的缺点}
xxxxxxxxxxxxxxxxxxxxxxxxxxxxxx


	\subsection{模型的改进与展望}%
xxxxxxxxxxxxxxxxxxxxxxxxxxxx
	\newpage	%换页符
	%%参考文献
	%\begin{thebibliography}{9}%宽度9
	% \setlength{\itemsep}{-2mm}
	
	\printbibliography[title = {参考文献}]	%使用国标参考文献添加方式
	%参考文献添加到wenxian.bib里,再引用
	
	\newpage
	%附录
	\appendix %%附录
\section{代码}
\subsection{爬取数据--python源代码}
\begin{lstlisting}[language=python]%这里修改语言
xxxxxxxxxxxxxxxxxxxxxxxxxxxxxxxx
\end{lstlisting}

\end{document}