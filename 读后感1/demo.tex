\documentclass{whutmod}
\usepackage{metalogo}
\usepackage{float}
\usepackage{subfigure} 
\usepackage{url}
\usepackage[style=caspervector,backend=biber,utf8]{biblatex}
\addbibresource{wenxian.bib}
\team{23}	% 组号
\membera{刘子川}
\joba{编程}
\memberb{程宇}
\jobb{建模}
\memberc{陈荣兴}
\jobc{建模}
\hypersetup{
	colorlinks=true,
	linkcolor=black
}
           
\title{论文《基于聚类分析的双目标优化定价模型》读后感}
%\tihao{3} % 题号

\begin{document}
	
	\maketitle
\Large   
我组读的论文题目为《基于聚类分析的双目标优化定价模型》,这是一篇由戴澄洁等(华中科技大学,2017B题高教社杯获得者)编写的一篇高教社杯全国大学生数学建模竞赛论文。

“拍照赚钱”APP 是基于移动互联网的自助式劳务众包平台,该篇论文通过建立数学模型,就 APP 中的任务定价问题进行分析,给出最优的任务定价方案并对模型进行评价。该论文数学模型醒目、图文详实、文献综述丰富而规范,其中论文关于双目标优化定价的不同方面都具有一定高度的见解。接下来将在几个方面谈谈自己的收获:

\section{文章大揽}


该篇论文在摘要部分的亮点主要有三个:一是全文重点使用了\textbf{黑体}标识,将关键词和具体结果进行描黑;二是每个问题描述思路清晰,在问题综述、预备工作、模型建立、问题结果等各个环节都有承上启下的论证;三是摘要后一段表明模型的优缺点分析,在论文中有详明阐述,吸引读者阅读兴趣。

在文体部分,图标展示是很特别具有代表性的。该文在每一个问题分析后都会画有思维流程 图,将模型概述方法进行图像化,让读者更容易读懂。其文章中更是有大量图表,清洗直观的可以观察到各个任务的位置信息和具体结果,这是特别值得学习的。其格式相当的符合学术规范,反映了作者很强的写作能力。

\section{模型方法总结}
该篇论文主要采用聚类分析的方法对论文中双目标优化函数进行处理,并遍历求解得到最优定价方案。最后利用BP神经网络对第四问进行了预测分析,具有一定的新颖性。本组认为,该文对优化定价模型进行了多角度的全面考查,并对优化函数进行多角度的约束限定,使得模型具有一定合理性。

针对问题一,该论文对项目的任务定价规律进行定性与定量研究。首先绘制出任务的经纬度坐标与定价数据的三维拟合图,并观测其分布密集的地区任务定价较低;再对位置数据进行空间离散化处理和 $K-Means$ 聚类分析。该论文对任务所处位置的周围环境进行确定影响因子时,包括了网格内任务数量、会员人数、会员平均完成能力、任务与中心点距离。再运用灰色关联矩阵定量分析,得到四个影响因子与定价的关系。

针对问题二,该论文设计设计了一个双优化模型,以总成本最小化、完成率最大化作为两个优化目标。通过问题一中任务未完成的原因分析引入吸引度矩阵,计算吸引力阈值。在目标函数上进行四个约束条件的确定:最大吸引准则,竞争准则,分配准则和时间列准则。利用深度多重搜索算法对决策变量进行遍历,得到最优定价方案。

针对问题三,在位置较为集中的任务被联合打包发布的情况下修改双目标定价优化模型。该论文根据任务的位置信息,进行多层次的聚类分析,并修改吸引力矩阵,重新计算得到每个任务的阈值。在满足问题二中的双目标优化模型的约束条件的情况下,多重搜索决策变量进行遍历,得到最优定价方案相对于第二问更为优化。

针对问题四,在基于双目标优化模型基础上聚类分析,得到新任务的打包方案。并利用第三问作为训练集建立 BP 神经网络预测模型,新任务的定价依旧满足问题二中的约束条件,通过预测得到新任务定价方案以及相应完成情况,最终定价都优于问题二和问题三的最优定价方案。

\section{模型的分析}

该文利用空间数据离散化的思想,将任务分布的经纬度区间划分为等面积的网格区域,便于统计每一个网格区间内的任务与会员的相关数据,计算影响因子的数值。在设立模型约束条件时,不仅从企业的目标优化角度考虑,同时考虑到每位会员不同的信誉值以及与之对应的预定开始时间和预定限额,模型的建立符合任务被预定的实际情形。不过此方法的缺点在于吸引度矩阵中元素的计算公式以及吸引度阈值的确定存在一定的主观性,需要获取的数据量十分巨大,给模型的求解与算法的实现带来了相当大的困难。

文中多次从定性与定量的两个角度分析任务的定价规律,通过图像直观清晰地观察定价 的定性规律,再定量给出影响因子对定价规律的影响程度。并且在问题三中进行多层次的聚类分析,运用二重嵌套聚类分析的方法,想法大胆又合理,保证打包方案的全面性与可行性。但各影响因子之间不可避免地存在相关性,未能充分考虑到影响因子之间的交互作用,对定价方案的设计存在一定影响,具有一定的优化空间。 


\section{结语}
这篇论文的结构严谨,层次分明,采用了递进式的分析结构,逻辑性强,表达清晰,是一篇很有价值文章。读过该论文后,我们不仅了解了如何更好的利用聚类分析与优化模型的建立,而且也学习到了一篇经典论文的脉络结构应该如何组织,值得我们深入探讨与学习。
\end{document}