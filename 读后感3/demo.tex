\documentclass{whutmod}
\usepackage{metalogo}
\usepackage{float}
\usepackage{subfigure} 
\usepackage{url}
\usepackage[style=caspervector,backend=biber,utf8]{biblatex}
\addbibresource{wenxian.bib}
\team{23}	% 组号
\membera{刘子川}
\joba{编程}
\memberb{程宇}
\jobb{建模}
\memberc{陈荣兴}
\jobc{建模}
\hypersetup{
	colorlinks=true,
	linkcolor=black
}
           
           
\title{论文《基0-1规划的单RGV动态调度模型》读后感}
%\tihao{3} % 题号

\begin{document}
	
	\maketitle
	\Large   
	我组学习的论文题目为《基于$0-1$规划的单RGV动态调度模型》,这是一篇2018年全国大学生数学建模国赛B题优秀论文。
	
	在自动化物流系统与自动化仓库中$RGV$的动态调度的研究对车间的高效作业有着重要意义。该篇论文通过将$RGV$调度模型转化为$0-1$规划模型,对$RGV$的工作模式进行仿真模拟,通过启发式算法分别优化三组$RGV$的行动路线并比对其结果。该论文数学模型合理、具有详细的计算推导过程与明晰易懂的模型规律。其中合理的模型假设简化了$RGV$的具体动作,提高了算法的计算效率。接下来对文章的各方面谈谈自己的理解和感受:
	
	\section{文章大揽}
	该论文在摘要部分的优点主要有三个:一是段落中的重点使用了\textbf{黑体}标识,将使用的模型方法和优化结果加粗描黑,加深了读者对模型所用方法和对应求解结果的印象;二是问题之间联系紧密,对问题的描述思路清晰,在问题综述、模型建立、问题结果等各个环节都有承上启下的论证;三是摘要整体构成总分结构,使文章层次清晰,逻辑明了。
	
	在文章主体部分,将$RGV$动态调度转化为$0-1$整数规划,通过一个简单的决策变量解决了复杂的路线规划,在模型建立中一步步将题目所给的条件转化为具有表达式的约束条件,并配合大量的流程图对算法予以说明,使文字的表述更加形象化,加深读者的理解。在模型求解中通过启发式算法与循环遍历法求解模型,并带入实际参数进行仿真检验以验证算法合理性。通过阅读代码后,作者能每一步都选个最优的解进行叠加,可看出作者的编码能力是非常值得我们学习的。
	
	
	\section{模型方法总结}
	该论文先分别在$CNC$有无发生故障的情况下分别对加工一道工序、两道工序的物料建立了$4$种$RGV$动态调度模型。该模块下$4$种模型均在第一类$RGV$的基础上建立$0-1$整数规划模型,并通过启发式算法求解。最后利用题给$3$组参数检验模型的实用性与算法的有效性。模型中所有约束条件都来自于提给智能加工规则,推导过程具逻辑清晰,使模型有较强可行性。

	~\\	
	\textbf{针对任务一:}
	
	关于情况$(1)$该论文将每轮作业$RGV$与$8$台$CNC$的关系用$0-1$逻辑变量表示,以所获成料数量最大作为目标函数,$RGV$的调度路径作为决策变量,智能加工系统中加工规则作为约束条件,结合$0-1$规划思想得到一道工序加工需求下的$RGV$动态调度模型,采用启发式算法对其求解,最终得到无故障时一道工序加工下的$RGV$的最优行动路径。
	
	关于情况$(2)$中具有两道工序的物料,在加工过程中有生料、已加工一道工序的物料以及熟料三种工序状态。在情况$(1)$的基础上引入了工序状态因子,并增加了总加工时间作为目标函数,建立了以$RGV$调度路径为决策变量的最快获得最大数量成料的双目标规划模型,最终通过循环遍历得出无故障时两道工序加工下的$RGV$的最优行动路径。
	
	
	关于情况$(3)$,当$CNC$存在$1\%$的概率故障时,在情况(1)、(2)所建立的规划模型下,该文将故障$CNC$的修理时间视为一次较长的加工时间,故障$CNC$在故障排除前不会发出需求指令。视工作人员排除故障的时间在$10-20$分钟间服从均匀分布,故障排除后$CNC$重启,此时报废物料已被移去$CNC$即刻发出需求指令。得到在$CNC$概率故障情况下的$RGV$动态规划模型,再循环遍历法得出存在故障时的$RGV$的最优行动路径。
	
	~\\
	\textbf{针对任务二:}
	
	拟考虑将模型的实用性分解为模型的普适性、经济性以及实施模型的可行性,$CNC$平均非有效工作时长,将算法的有效性分解为程序总运行时长、Matlab的内存占用量。利用三组系统参数分别进行仿真验证,讨论计算过程与结果的模型的实用性与算法的有效性。
	
	\section{模型的分析}
	该论文从$0-1$规划模型出发,规律简单、易懂,而且能够运用该模型以及模型求解算法得出比较理想的调度方案,说明了模型的实用性和算法有效性。在模型$1$的基础上加入了目标工序的概念,以此来保证$RGV$中的物料工序与$CNC$工序相匹配。添加约束条件与目标函数得到双道工序的$RGV$调度模型,对于模型的求解采用循环遍历法,能够找到各刀具数量与位置的最佳分配方案。在模型 1、2 的前提下,建立故障模拟模型,能够较好地表示出$1\%$的故障发生概率,而且能够及时发现与处理故障 CNC,提高了工作的效率,从而能找出更优的调度方案提高了模型的实用性。该模型存在缺点即问题一中的启发式算法实际为贪婪算法,求得解可能并不是全局最优解。
	
	
	\section{结语}
	整篇论文的结构合理,层次分明,模型推导过程中采用了递进式的分析结构,科学严谨,逻辑性强,表达清晰,是一篇很有价值的文章。读过该论文后,我们不仅了解了$0-1$整数规划和动态规划,而且也学习到了模型建立中科学严谨的求解过程,以及合理的模型假设对于问题求解的极大帮助,这都值得我们进一步深入探讨与学习。
\end{document}