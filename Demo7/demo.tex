\documentclass{whutmod}
\usepackage{metalogo}
\usepackage{float}
\usepackage{subfigure} 
\usepackage{url}
\usepackage{booktabs}
\bibliographystyle{unsrt}
\team{23}
\membera{刘子川}
\joba{编程}
\memberb{程宇}
\jobb{建模}
\memberc{陈荣兴}
\jobc{建模}
\hypersetup{
	colorlinks=true,
	linkcolor=black
}

\title{基于因子分析与灰色关联分析对武汉市人才吸引力的量化评价}
\tihao{4} 

\begin{document}


	
	\begin{abstract}



本文基于多背包优化模型,设计\textbf{沙石算法},优化部队的装载方案。再通过\textbf{遗传算法}与\textbf{泊口仿真模型},选择出装载方案中装载时间最少,需要船支数量最少的装载方案。最后在\textbf{泊口仿真模型}中添加港口摧毁事件,再次使用使用遗传算法求解出港口被摧毁对应的最佳装载方案。
~\\

针对问题一,将原问题转化成多背包优化问题,设计\textbf{沙石算法},优化部队装载方案。根据题目要求,首先对兵力、装备与船载的面积进行量化,得到相关的有效面积值。然后建立船舰数量的决策向量,以调用船支数量作为目标函数,题中船舰装载限制作为约束条件,建立\textbf{部队装载优化模型}。设计\textbf{沙石算法},进行装备均分和部队装载,求解得到各旅级单位装备人口的装载方案和各类舰船的使用数量。各类船舰的有效面积利用率基本达到$\textbf{90\%}$以上,共使用船舰数\textbf{236艘}。
~\\



针对问题二,调整问题一中的\textbf{沙石算法}作为验证算法,并搭建\textbf{泊口仿真模型}计算每种装载方案需要的装载时间作为适应度函数,代入\textbf{遗传算法}求得最佳装载方案。本组首先基于问题一中设计的\textbf{沙石算法},编译验证算法,以验证每种装载方案是否能装载全体部队。然后设计\textbf{泊口仿真模型},使每个港口以最大工作效率工作,计算用时最长泊口装载所用时间作为总装载时间。再利用遗传算法,将每种民用船调用量作为染色体,搜素总装载时间最少,使用总船支数量最少的调用方案。最优方案为军用登陆舰\textbf{全部调用},民用船调用方案为2万吨级滚装船\textbf{15艘},3万吨级滚装船\textbf{5艘},集装箱船\textbf{1艘},杂货船\textbf{1艘}和客船\textbf{1艘}。装载时间为\textbf{20小时},共使用船舰\textbf{364艘}。
~\\

针对问题三,基于问题二中的模型,延用遗传算法,并调整\textbf{泊口仿真模型},增加摧毁泊口事件,求解出泊口被摧毁对应的最佳装载方案。首先调整\textbf{泊口仿真模型},当运行时间等于攻击时间时,停止被摧毁港口正在进行的任务,并将任务进度清零,并在之后的检索列表中剔除被摧毁泊口。重新运行遗传算法以寻找港口被摧毁对应的最佳装载方案。最优方案为军用登陆舰\textbf{全部调用},民用船调用方案为2万吨级滚装船\textbf{18艘},3万吨级滚装船\textbf{3艘},集装箱船\textbf{1艘},杂货船\textbf{1艘}和客船\textbf{1艘}。装载时间为\textbf{20小时},共使用船舰\textbf{365艘}。
~\\

本文中所提到的模型优点主要有两点:一、使用沙石算法与泊口仿真模型,使得军舰与泊口的利用率达到较高水平,最终方案转载时间少,使用船支数量少;二、利用遗传算法优化装载方案,鲁棒性强,全局搜素能力强。

	
  
\keywords{沙石算法\quad  遗传算法\quad  泊口仿真模型\quad 部队装载优化模型\quad }



	\end{abstract}
	
	%目录
	\tableofcontents
	\newpage	%换页符
	
	\section{问题重述}	
	\subsection{问题背景}
    随着全球化的进程,人类活动范围日益扩大,人群流动频繁,传染病可在大范围内迅速传播,是对人类社会存在威胁的公共卫生问题。在疾病控制实际工作中,疾病的发病与流行趋势分析是极其重要的一环,科学、准确的分析能对卫生行政部分制定疾病预防与控制策略产生重要的影响,传染病早期预警将大大降低传染病的社会经济危害。
    
    为了提高某传染病疫情和突发公共卫生事件报告的质量和时效,加强对全国感染病人的诊断、治疗和督导管理,卫生部建立了全国监管机制,及时通报相关病情和相关数据,并通过对疫情数据的动态分析,建立该传染病防治工作督导检查、防治效果评价和制定防治对策和策略,控制并逐渐消灭该传染病。构建预测模型从早期探测到传染病的爆发并及时预警,采取应对措施,是目前传染病防控的重要手段,具有重要的实际意义。
    
    

	\subsection{问题概述}
    围绕相关附件和条件要求,研究海运装载行动输送兵力任务的合理安排,依次提出以下问题:
		 
	
	\textbf{问题一:}根据合适的指标建立模型,分析流行病在2004-2016年的变化趋势,并预测2019年全国感染该病的发病数和死亡数。
	
	\textbf{问题二:}基于2004-2016年每隔三年的不同地区的和职业分类的数据,建立疾病传播模型,并预测2019年传染病重点防控前3名的区域和职业人群。
		
	\textbf{问题三:}结合地区经济发展的相关数据,选择一个角度建立传染病与经济发展相关的模型,并分析结论。
	
	\textbf{问题四: }综合模型结果及分析,给卫生健康委员会相关部门写一封公开信,谈谈对传染病疫情防治的看法和建议。
	
	
	\section{模型假设}
	\begin{itemize}                                             
		\item [(1)] 为保证预测结果精确性,假设题目所给出数据真实可信。
		\item [(2)] 为了优化运算结果,假设所有全副武装的士兵都保持坐姿休息。
		\item [(3)] 假设在战争中,装载消耗更少时间的优先度高于使用更少的民用船。
		\item [(4)] 假设在装载过程中,同一泊口的船舰装载交替时间可以忽略不记。
		\item [(5)] 假设港口被摧毁时,由于提前得到信息,港口上的船只与兵力没有损失,只是正在进行的装载工作停止,且进度完全损失。
	\end{itemize}
	
	
	\section{符号说明}
	\begin{table}[H]
	\label{biao} \centering

	\begin{tabular}{cc}
		\toprule[1.5pt]
		\multicolumn{1}{m{5cm}}{\centering 符号} & \multicolumn{1}{m{5cm}}{\centering 说明} \\
		\midrule[1pt]		
		$X_{i}$  &  第i型装备 \\ 
		$s_{xi}$  &  装备$X_{i}$的占用面积 \\ 
		$l_{i}$  & 装备$X_{i}$的长\\
		$w_{i}$  &  装备$X_{i}$的宽 \\ 
		$\varepsilon _{i}$ & 装备$X_{i}$的面积修正系数\\
		$p_{i}$	 &  全副武装人员数  \\ 
		$eta_{1}$ &  海军船舰的有效面积率 \\ 
		$eta_{2}$	 &  民用船只的有效面积率 \\ 
		$y_{k}$  &   登陆舰数量\\ 
		$z_{n}$  &  民用船数量\\	
		$D$ & 决策向量\\
		$S_{D}$ &  舰载面积系数向量\\ 
		$S_{z}$ & 民用船有效面积向量\\
		$S_{y}$ & 登陆舰有效面积向量\\
		$S_{p}$ & 每营人口面积向量\\
		$A^{k}$  & 面积规划向量\\
		$T_{n}$ & 颖编制总部对数列\\
		$P_{n}$ & 船舰面积数列\\
		$\eta $ &  平均有效面积利用率\\

		\bottomrule[1.5pt]
	\end{tabular}
\end{table}

	\section{问题一模型的建立与求解}
    \subsection{问题描述与分析}

    问题一要求,根据附件中2004年至2016年的流行病相关数据,预测2019年全国感染该疾病的发病人数和死亡人数。本组首先基于灰色预测,预测分析2004-2019年该流行病的发病人数和死亡人数。再通过马尔科夫模型,由2004-2016年的数据模拟残差在各个区间的分布,计算2017-2019年预测残差的期望值。最后将预测结果与残差期望做差,校正传统灰色预测的固有偏差。
   
	    \subsection{模型的建立}
	    \subsubsection{灰度预测GM(1,1)}
	    设2004-2016年总发病人数为时间序列:
	     \begin{gather}
	    X^{(0)}=[x^{(0)}(1),x^{(0)}(2),\cdots,x^{(0)}(13)]
	    \end{gather}
	    通过一次累加生成1-AGO序列:
	    \begin{gather}
	    X^{(1)}=[x^{(1)}(1),x^{(1)}(2),\cdots,x^{(1)}(13)]
	    \end{gather}
	    式中:$x^{(1)}(k)=\sum_{i=1}^{k}x^{(1)}(i),k=1,2,\cdots,13$。根据1-AGO序列建立微分方程为:
	     \begin{gather}\label{333}
	    \frac{d X^{(1)}}{dt}+a X^{(1)} = u
	     \end{gather}
	     其中,$a$称为发展灰度,$u$称为内生控制灰度。设$\widehat{\alpha }$为待估参数向量,且$\widehat{\alpha }=[a,u]^T$,利用最小二乘法求出:
	     \begin{gather}
	     \widehat{\alpha }=(B^TB)^{-1}B^{T}Y_{n}
	     \end{gather}
	     求解方程(~\ref{333}~),可得第$k+1$年能源消费预测为:
	     \begin{gather}
	     \widehat{X}(k+1)=[X^{(0)}(1)-\frac{u}{a}]e^{-ak}+\frac{u}{a},k=1,2,\cdots,16
	     \end{gather}
	     同理将死亡数作为向量$X^{(0)}=[x^{(0)}(1),x^{(0)}(2),\cdots,x^{(0)}(13)]$带入模型可求得2017-2019年死亡数灰度预测值。
	     \subsubsection{马尔科夫模型校正}
	     利用马尔科夫模型对GM(1,1)预测误差项的状态及状态概率进行预估,并利用预测状态的期望值对GM(1,1)预测值进行修正。用2004-2016年预测数据与真实数据残差进行状态划分,设残差序列为:
	     \begin{gather}
	    \varepsilon =[\varepsilon(1) ,\varepsilon(2), \cdots,\varepsilon(13)]
	     \end{gather}
	    最大残差绝对值为$\delta _{max}=\underset{1\leqslant i\leqslant13 }{max}\left | \varepsilon(i) \right |$,将预测误差化均分为三个状态。令$\lambda =\frac{\delta _{max}}{6}$。状态分别为$E_{1}:(-3\lambda,-\lambda)$、$E_{2}:(-\lambda,\lambda)$和$E_{1}:(\lambda,3\lambda)$。其中初始状态概率向量计算公式为:
	   \begin{gather}
	 t_{0}=[p_{E1},p_{E2},P_{E3}]\\
	 p_{Ek}=\frac{n_{Ek}}{13}
	  \end{gather}
	  式中$n_{Ek}$是状态$E_{k}$在2004-2016年内出现的次数,以状态$E_{k}$出现的频率代替其出现的概率$p_{Ek}$。且构建状态转移矩阵为:
	   \begin{gather}
	  P=\left(\begin{array}{lll}{P_{11}} & {P_{12}} & {P_{13}} \\ {P_{21}} & {P_{22}} & {P_{23}} \\ {P_{31}} & {P_{32}} & {P_{33}}\end{array}\right)
	  \end{gather}
	  其中$P_{ij}$是由状态$E_{i}$经过一个时期转移到$E_{j}$的转移概率。即马尔科夫模型可表示为:	  
	   \begin{gather}
	   t_{k+1}=t_{k} \cdot p
	  \end{gather}
	  设状态区间的中间值分别为$\overline{E}_{1}$、$\overline{E}_{2}$和$\overline{E}_{3}$,即第k年GM(1,1)的误差期望为:
	   \begin{gather}
\eta =\begin{bmatrix}
p_{E1} & p_{E2} & p_{E3}
\end{bmatrix} \cdot\begin{bmatrix}
\overline{E}_{1}\\ 
\overline{E}_{2}\\ 
\overline{E}_{3}
\end{bmatrix}
	  \end{gather}
	  设第$k$年的患病人数的GM(1,1)预测值为$\widehat{x}(k)$,修正后的组合预测值$\overline{x}(k)$可以记作:
	     \begin{gather}
	     \overline{x}(k) =\widehat{x}(k)-\eta
	   \end{gather}
	
	\subsection{模型的求解}   
	  
	  
	  
	  \begin{table}[H]
	  	\centering\caption{相对误差和状态区间}\label{qujian}
	  	\begin{tabular}{cccccccccccccc}
	  		\toprule[2pt]
	  		\multicolumn{1}{m{2cm}}{\centering 年份}
	  		& \multicolumn{1}{m{.7cm}}{\centering 2004}
	  		&\multicolumn{1}{m{.7cm}}{\centering 2005}
	  		& \multicolumn{1}{m{.7cm}}{\centering 2006}
	  		& \multicolumn{1}{m{.7cm}}{\centering 2007}
	  		& \multicolumn{1}{m{.7cm}}{\centering 2008}
	  		& \multicolumn{1}{m{.7cm}}{\centering 2009}
	  		& \multicolumn{1}{m{.7cm}}{\centering 2010}
	  		& \multicolumn{1}{m{.7cm}}{\centering 2011}
	  		& \multicolumn{1}{m{.7cm}}{\centering 2012}
	  		& \multicolumn{1}{m{.7cm}}{\centering 2013}
	  		& \multicolumn{1}{m{.7cm}}{\centering 2014}
	  		& \multicolumn{1}{m{.7cm}}{\centering 2015}
	  		& \multicolumn{1}{m{.7cm}}{\centering 2016}
	  		\\
	  		\midrule[1pt]
	  		所处状态区间 &  $E_{2}$  &$E_{2}$ & $E_{1}$&$E_{2}$ &$E_{3}$ &$E_{2}$&$E_{1}$&$E_{1}$&$E_{2}$&$E_{2}$&$E_{2}$&$E_{2}$&$E_{2}$  \\ 
	  		\bottomrule[2pt]	
	  	\end{tabular}
	  \end{table}
	  

	    \begin{table}[H]
	    	\centering\caption{死亡数状态区间划分}\label{siwang}
	    	\begin{tabular}{ccccc}
	    		\toprule[2pt]
	    		 \multicolumn{1}{m{3cm}}{\centering 状态}
	    		& \multicolumn{1}{m{2cm}}{\centering $E_{1}$}
	    		& \multicolumn{1}{m{2cm}}{\centering $E_{2}$}
	    		& \multicolumn{1}{m{2cm}}{\centering $E_{3}$}
	    		\\
	    		\midrule[1pt]
	    		相对误差区间 &  $(-684,-228]$  &$(-228,228)$ & $(228,684]$   \\ 
	    		\bottomrule[2pt]	
	    	\end{tabular}
	    \end{table}
	  
	 \begin{table}[H]
		\centering\caption{生病数状态区间划分}\label{shengbing}
		\begin{tabular}{ccccc}
			\toprule[2pt]
			\multicolumn{1}{m{3cm}}{\centering 状态}
			& \multicolumn{1}{m{2cm}}{\centering $E_{1}$}
			& \multicolumn{1}{m{2cm}}{\centering $E_{2}$}
			& \multicolumn{1}{m{2cm}}{\centering $E_{3}$}
			\\
			\midrule[1pt]
			相对误差区间 &  $(-66389,-22130]$  &$(-22130,22130)$ & $(22130,66389]$   \\ 
			\bottomrule[2pt]	
		\end{tabular}
	\end{table}  
	  
	  
	  
	
		
	\section{问题二模型的建立与求解}
	\subsection{问题的描述与分析}

	问题二要求针对具体运输任务,结合港口泊位,制定时间最短、船只数量最少的兵力装载方案。本组通过\textbf{遗传算法},将问题一中的决策向量作为染色体,装载时间和船支数量作为适应度函数,建立装载方案优化模型。修改问题一中的\textbf{沙石算法},作为验证算法判断每个个体是否可以存活,并随机生成$w$个能通过验证算法的初始个体。将某一代个体经过交叉、变异后的基因带入\textbf{时间轴仿真模型},计算出装载所用时间作为适应度函数,将生成个体按照适应度函数大小进行第一次排序,再将装载所用时间大小相同的函数进行第二次排序,取出排序中前$w$个个体进行下一次进化,进化$g$代个体后可求得\textbf{装载时间最短、使用海上运输工具最少}的兵力装载方案。	                                                                                                   
	\subsection{模型的建立}
	    \subsubsection{优化函数与约束条件}
	基于问题一模型\textbf{决策向量}为:
	\begin{gather*}
	D=[y_{1},y_{2},\cdots,y_{14},z_{1},z_{2},\cdots,z_{5}]
	\end{gather*}
	\textbf{目标函数}为:
	\begin{gather}
	min Z=\sum _{1}^{19}D[k]
	\end{gather}
	增加\textbf{目标函数}:
		\begin{gather}
	min \left \{ \underset{k}{max}\sum t_{i} \right \}
	\end{gather}
	其中$\sum t_{i}(i=1,2\cdots,71)$,表示在共计$71$个泊口中第$i$个泊口的装载时间总和。	
	

	且由于集装箱船(Z3)无法装载装备X1,X2,X3和X4,增加	\textbf{约束条件}:
	\begin{gather}
	D \cdot S_{D} - z_{3} \cdot s_{z3} \geq  \sum s_{x1}+ \sum s_{x2}+\sum s_{x3}+ \sum s_{x4}
	\end{gather}
	
	
		综上,建立得到问题二\textbf{时间与船数优化模型}:

	\begin{gather}
min Z=\sum _{1}^{19}D[k]\\
min \left \{ \underset{k}{max}\sum t_{i} \right \}\\
s.t.\left\{\begin{matrix}	 D\cdot S_{D}\geq \sum  S_{x} + \sum  S_{p}
\\ y_{1}\cdot s_{y1}\geq \sum s_{x9}+ \sum  s_{p6}
\\S_{D}=[S_{y}, S_{z}]
\\ s_{xi}=l_{i}\times w_{i} \times \varepsilon _{i}
\\S_{x}=[s_{x1},s_{x2},\cdots,s_{x14}]
\\s_{pi}=p_{i}\times s
\\S_{p}=[s_{p1},s_{p2},\cdots,s_{p12}]
\\     s_{yi}=s_{i}\times \eta_{1}
\\   S_{y}=[s_{y1},s_{y2},\cdots,s_{y14}]
\\      s_{zj}=s_{j}\times \eta_{2}
\\ S_{z}=[s_{z1},s_{z2},\cdots,s_{z5}]
\\	D \cdot S_{D} - z_{3} \cdot s_{z3} \geq  \sum s_{x1}+ \sum s_{x2}+\sum s_{x3}+ \sum s_{x4}
\end{matrix}\right. 
\end{gather}
	     ~\\
	其中$ \underset{k}{max}\sum t_{i} $可利用\textbf{泊口仿真模型}计算,并通过\textbf{遗传算法}算得全局最优解。
	\subsubsection{泊口仿真模型}
	建立泊口仿真模型以计算每种方案的装载时间,首先输入决策向量:

		\begin{gather*}
	 D=[y_{1},y_{2},\cdots,y_{14},z_{1},z_{2},\cdots,z_{5}]
		\end{gather*}
	定义时间函数为$t_{0}$,初始化时间函数令$t_{0}=0$,并令$t_{0}$以$1$为步长逐渐增加。
	定义港口决策向量为:
		\begin{gather*}
		T=[t_{1},t_{2},\cdots,t_{71}]
		\end{gather*}
	$t_{k}(k=1,2,3,\cdots,71)$代表$71$个港口当前任务的结束时间。当$t_{k}=t_{0}$ 时,表示港口$k$处于空闲状态,此时立即更新任务结束时间$t_{k}$,即使得:
		\begin{gather*}
		t_{k}=t_{k}+t_{d}\\
		D[d]=D[d]-1
		\end{gather*}
	其中$t_{d}$是决策向量$D$中第$d$各元素对应船舰的装载时间,$D[d]$为决策向量$D$中的第$d$个元素。当
		\begin{gather*}
	\left\{\begin{matrix} \sum_{1}^{19}D[d]=0
	\\t_{0}\geqslant max t_{k} 	
	\end{matrix}\right.
		\end{gather*}

		即所有船支装载完成时,结束运算,输出当前的时间函数$t_{0}$,即:
			\begin{gather*}
		min \left \{ \underset{k}{max}\sum t_{i} \right \}=t_{0}
			\end{gather*}
    \begin{figure}[H]
	\centering
	\includegraphics[width=.9\textwidth]{figures/yyy.png}
	\caption{泊口仿真流程图}\label{dieyyrwen}
\end{figure}
	\subsubsection{遗传算法}
	 \paragraph{编码}
	 如图所示,每个个体由下面三个方块构成,其中最后一个进行遗传操作:
	 	\begin{figure}[H]
	 	\centering
	 	\includegraphics[width=.5\textwidth]{figures/yichuan.jpg}
	 	\caption{指标重要度占权图}\label{yichuan}
	 	 \end{figure}
	 根据题目要求,优先使用军用登陆舰,即决策向量$D=[y_{1},\cdots,y_{14},z_{1},\cdots,z_{5}]$中$y_{k}(k=1,2,\cdots,14)$为恒定值且等于其上限。故能简化决策向量为:
	\begin{gather*}
	D=[z_{1},z_{2}\cdots,z_{5}]
	\end{gather*}
	 \paragraph{交叉}
	 为保证变异率并保留优秀基因片段和本题采用的两种交叉方式:
	\begin{itemize}
	\item [(1)]单点交叉:对于两个父代个体$D=[z_{1},z_{2}\cdots,z_{5}]$和	$D'=[z_{1}',z_{2}'\cdots,z_{5}']$,随机选择第$k$个基因处为交叉点,将该基因后所有基因进行交换,得到子代基因
	\item [(2)]中间值交叉:对于两个父代个体$D=[z_{1},z_{2}\cdots,z_{5}]$和	$D'=[z_{1}',z_{2}'\cdots,z_{5}']$,随机选取$z_{k}''\in [z_{k},z_{k}']$得到子代基因。
    \end{itemize}
    再选取交叉个体时采用混合分组的方法,将父代均匀混合后选取所有编号为奇数的个体,与其相邻对应编号为偶数的个体,通过两种交叉方式产生处两种类型的子代。
     \paragraph{变异}
     为保证种群多样性,以$0.1$的变异率,对选择的个体执行变异操作。随机选择变异个体中的基因$z_{k}$,使其值以各$50\%$的概率加一或减一。
     \paragraph{筛重}
     由于该模型基因维度较低,有大概率出现基因重复的个体,其对种群多样性有不利影响,并易使算法早熟。故在每次交叉变异生成新个体时删除重复基因,以提高算法的全局搜素能力。
     \paragraph{选择}
     由于本题约束条件较多,且有两个包含两个适应度函数,故采用两种选择方式:
     \begin{itemize}
     \item [(1)] \textbf{约束淘汰:}对于生成的个体基因$D=[z_{1},z_{2}\cdots,z_{5}]$带入第一问的沙石算法中,计算部队是否可以完全装入船支中。若不能则删除个体,若能则保留个体。
     \item [(2)]\textbf{精英选择:}由于模型假设战场上使用较少的时间优先级高于使用更少船支,将每一代中的生成个体与上一代混合,的个体按照其\textbf{装载时间}由大到小进行\textbf{第一次排序},再将装载时间相同的个体进行依照\textbf{调用总数}进行\textbf{第二次排序}。保留前列的$w$个个体,$w$为设定的种群容纳量。
     \end{itemize}
     重复上述进化过程,当进化代数足够多时,求解得到全局最优解。

     \subsection{模型求解与算法实现}


   	 (1) 将染色体中的单个基因取出,带入问题一沙石算法中。计算只使用的某种民用船时,需要的该民用船的数量,作为单个基因的取值范围。在使用卡特蒙洛法随机生成初始个体,当生成$50$个能通过沙石算法检验的初始个体时,停止生成。
   	 
   	 (2)将输入体进行、交叉、变异、筛重和约束淘汰后,得到子代生成个体。将子代个体与附带个体混合,带入泊口仿真模型求出每个个体的装载时间作为适应度函数。根据装载时间对每个个体进行第一次排序,并将装载时间相同的个体进行第二次排序,取序列中的前$50$个个体作为下一代输入个体。

   (3)重复上述步骤(2),当进化次数达到$100$次时结束进化,选取当前最优个体作为输出的个体,将输出个体基因与军用登陆舰总调遣数量合并即可得到完整的船支调遣方案。


    	 
    	 其求解该模型的具体流程图如~\ref{dierwen}~所示:
    \begin{figure}[H]
    	\centering
    	\includegraphics[width=\textwidth]{figures/dierwen.png}
    	\caption{沙石遗传算法与时间轴仿真流程图}\label{dierwen}
    \end{figure}
     
     	\subsection{结果分析}
     	遗传算法的算法收敛图如图~\ref{sdsd}~所示:
     	    \begin{figure}[H]
     		\centering
     		\includegraphics[width=.7\textwidth]{figures/shou1.png}
     		\caption{遗传算法收敛图}\label{sdsd}
     	\end{figure}
     	
     	由算法收敛图可知算法收敛速度较慢,早熟问题解决的较好,全局搜素能力较强,求得最优个体基因为:$[15,5,1,1,1]$
     	
     	其对应装载方案为军用登陆舰全部调用,民用船调用方案为2万吨级滚装船$15$艘,3万吨级滚装船$5$艘,集装箱船$1$艘,杂货船$1$艘和客船$1$艘,共使用船舰$23$艘。
     	其对应的泊口装载时间图如图~\ref{sbsb}~所示:
     	\begin{figure}[H]
     		\centering
     		\includegraphics[width=\textwidth]{figures/bo1.png}
     		\caption{泊口装载时间图}\label{sbsb}
     	\end{figure}
     	
     	其中最后完成装载的泊口用时$20$个小时。即最优装载方案中装载时间为$20$个小时。
     	
     \section{问题三模型的建立与求解}
   	\subsection{问题的描述与分析}
   	问题三要求,当装载行动开始24小时后,A港和D港有部分泊位被毁,给出调整方案和装载建议。
   	本组基于问题二中模型求解的结果,作出合理假设,假设泊位被毁不损失船只和兵力,由于问题二中的装载时间最优解小于$24$小时,根据建议将码头被摧毁时间改为第$18$小时。本组延用问题二中的遗传算法,当时间轴进行到$18$小时,剔除已经完成装载任务的船舰和A港、D港中部分被毁的码头泊位,修改问题二中的港口仿真模型。通过遗传算法求解出对剩余船舰、码头泊位的装载方案,给出装载建议。
  \subsection{模型的建立}
    \subsubsection{模型调整}
    \textbf{优化函数与约束条件}延用问题二模型:
     	\begin{gather}
min Z=\sum _{1}^{19}D[k]\\
min \left \{ \underset{k}{max}\sum t_{i} \right \}\\
s.t.\left\{\begin{matrix}	 D\cdot S_{D}\geq \sum  S_{x} + \sum  S_{p}
\\ y_{1}\cdot s_{y1}\geq \sum s_{x9}+ \sum  s_{p6}
\\S_{D}=[S_{y}, S_{z}]
\\ s_{xi}=l_{i}\times w_{i} \times \varepsilon _{i}
\\S_{x}=[s_{x1},s_{x2},\cdots,s_{x14}]
\\s_{pi}=p_{i}\times s
\\S_{p}=[s_{p1},s_{p2},\cdots,s_{p12}]
\\     s_{yi}=s_{i}\times \eta_{1}
\\   S_{y}=[s_{y1},s_{y2},\cdots,s_{y14}]
\\      s_{zj}=s_{j}\times \eta_{2}
\\ S_{z}=[s_{z1},s_{z2},\cdots,s_{z5}]
\\	D \cdot S_{D} - z_{3} \cdot s_{z3} \geq  \sum s_{x1}+ \sum s_{x2}+\sum s_{x3}+ \sum s_{x4}
\end{matrix}\right. 
\end{gather}
     ~\\
     调整 \textbf{泊口仿真模型},延用港口决策向量:
     \begin{gather*}
     T=[t_{1},t_{2},\cdots,t_{71}]
     \end{gather*}
     当时间轴函数$t_{0}=18$时,停止被毁泊口正在进行的任务,使得:
     \begin{gather}
     \left\{\begin{matrix}
     t_{k}=-1(k=1,2,11,37,46,47)\\ 
     D[d]=D[d]+1
     \end{matrix}\right.
     \end{gather}
     即立刻终止港口正在进行的工作,且在$18$小时后检索港口决策向量时,剔除$t_{k}(k=1,2,11,37,46,47)$(其对应被摧毁的港口无法执行装载任务)。并将修改后泊口仿真模型算得的装载时间,带入遗传算法的适应度函数值,重新运行遗传算法。即可求得装载中途港口被摧毁后的最优船舰调整方案。
       \subsection{模型的求解}
     \paragraph{算法的实现}
     	沿用第二问算法,仅修改泊口仿真模型:如果达到摧毁时间,将第$i$个摧毁港口的任务结束时间$t_{ki}$修改为$-1$,并不再更新此港口。若此时港口$i$不是空闲港口,将对应船舰$d_{i}$的数目增加一个。其算法流程图如~\ref{dieyyrywen}~:
    \begin{figure}[H]
	\centering
	\includegraphics[width=\textwidth]{figures/yyyy.png}
	\caption{泊口仿真改进流程图}\label{dieyyrywen}
\end{figure}

	\subsection{结果分析}
     	求得最优个体基因为:$[15,5,1,1,1]$
     
     其对应装载方案为军用登陆舰全部调用,民用船调用方案为2万吨级滚装船$18$艘,3万吨级滚装船$3$艘,集装箱船$1$艘,杂货船$1$艘和客船$1$艘,共使用民用船舰$24$艘。
     其对应的泊口装载图如图~\ref{nb}~所示:
   	\begin{figure}[H]
     	\centering
     	\includegraphics[width=\textwidth]{figures/bo2.png}
     	\caption{泊口装载时间图}\label{nb}
     \end{figure}
     
     其中最后完成装载的泊口用时$20$个小时。即最优装载方案中装载时间为$20$个小时。



	\section{模型的评价}
	\subsection{模型的优点}
		\begin{itemize}                                             

		\item [(1)] 将多目标优化转换成背包优化问题,自主设计沙石算法,使得船舰的面积利用率高,对船只数量的需求量少。
		\item [(2)] 使用遗传算法,具有很强的全局搜索能力和鲁棒性,运算时间远小于全遍历算法。
	\end{itemize}
	\subsection{模型的缺点}

	遗传算法初始解由卡特蒙洛法随机生成,每次搜索结果不完全相同,可能引起结果的偏差,需要多搜索几次选择才能得到最优结果。
	\subsection{模型改进}
	可使用改进的生命遗传算法,加强算法的局部搜索能力,解决算法早熟的问题。

 
	\newpage	%换页符
	%%参考文献
	%\begin{thebibliography}{9}%宽度9
	% \setlength{\itemsep}{-2mm}
	\nocite{*}		%排版未引用的参考文献
%\bibliography{wenxian.bib}
%	%参考文献添加到wenxian.bib里,再引用
%	
\begin{thebibliography}{9}%宽度9
	\bibitem{bib:one}Febri Liantoni,Rifki Indra Perwira,Syahri Muharom,Riza Agung Firmansyah,Akhmad Fahruzi. Leaf classification with improved image feature based on the seven moment invariant[J]. Journal of Physics: Conference Series,2019,1175(1).

\end{thebibliography}

	\newpage
	%附录
	\appendix %%附录

\section{代码}
\subsection{问题一沙石算法--python源代码}
\begin{lstlisting}[language=python]
import math
import numpy as np
import pandas as pd


def append(list,x):
'''numpy数组中插入一个元素在末尾'''
return np.append(list,x)


def Merge(dict1, dict2):
'''字典合并'''
res = {**dict1, **dict2}
return res


def shell_sort(lists):
'''
按从大到小排序,时间复杂度O(n)
:param lists:传进来的数组,比如船只或者包裹
:return: 返回从大到小的包裹
'''
# 希尔排序
lists = sorted(lists,reverse=True)
return lists


def kv_sort(kv):
'''
键值对由大到小排序
eg:
d = {'d1':2, 'd2':4, 'd4':1,'d3':3,}
res = sorted(d.items(),key=lambda d:d[1],reverse=True)
[('d2', 4), ('d3', 3), ('d1', 2), ('d4', 1)]
:param kv:键值对
:return:元组
'''
res = sorted(kv.items(), key=lambda d: d[1], reverse=True)
return res


def min_index(lists):
'''
返回最穷连的下标
:param lists: 各连的装备面积数目
:return: 最穷的那个连的下标
'''
index_min = 0
for i in range(1, len(lists)):
if lists[i] < lists[index_min]:
index_min = i
return index_min


def ship_list2dict(sum_ship):
'''
船变字典
:param sum_ship:船的列表
:return: 船的字典
'''
# 船只变成列表
# 列表转字典
dict_ship = {}
for i in range(len(sum_ship)):
if i < 5:
dict_ship["Y1_" + str(i)] = sum_ship[i]
elif i < 8:
dict_ship["Y2_" + str(i - 5)] = sum_ship[i]
elif i < 17:
dict_ship["Y3_" + str(i - 8)] = sum_ship[i]
elif i < 21:
dict_ship["Y4_" + str(i - 17)] = sum_ship[i]
elif i < 32:
dict_ship["Y5_" + str(i - 21)] = sum_ship[i]
elif i < 42:
dict_ship["Y6_" + str(i - 32)] = sum_ship[i]
elif i < 43:
dict_ship["Y7_" + str(i - 42)] = sum_ship[i]
elif i < 44:
dict_ship["Y8_" + str(i - 43)] = sum_ship[i]
elif i < 75:
dict_ship["Y9_" + str(i - 44)] = sum_ship[i]
elif i < 87:
dict_ship["Y10_" + str(i - 75)] = sum_ship[i]
elif i < 97:
dict_ship["Y11_" + str(i - 87)] = sum_ship[i]
elif i < 297:
dict_ship["Y12_" + str(i - 97)] = sum_ship[i]
elif i < 333:
dict_ship["Y13_" + str(i - 297)] = sum_ship[i]
elif i < 341:
dict_ship["Y14_" + str(i - 333)] = sum_ship[i]
return dict_ship


def sum_ship_list(ship_S, ship_num):
'''
计算所有的船只数目/装备数目
:param ship_S: 船只面积/装备面积
:param ship_num: 各类船只数目/装备数目
:return: 所有用有船只面积/装备面积
'''
sum_ship = []
if(len(ship_num)==len(ship_S)):
for i in range(len(ship_num)):
for j in range(ship_num[i]):
sum_ship.append(ship_S[i])
else:
print('船数与船面积的两个列表不相等,可能是打错了\n')
return sum_ship


def zhuang_baoguo(zhuangbei_num_list, zhuangbei_S_list, lian_num, lian_mianji, type):
'''
给出装备数目,装备面积,要求*均分*给每个连
均分打包思想:谁穷我给谁装备:
装备我从小到大排序;
for 装备的数目:
装备先给连j;
判断哪个连最穷
给穷的连装备

:param zhuangbei_num_list: 各个装备的数目
:param zhuangbei_S_list: 不同装备的面积
:param lian_num: 连的数量
:param lian_mianji: 连中人的面积
:param type:   旅的类型,键
:return: 装备打包好了后,下一步就是上船
'''

#初始化连的数目
lian = [0.0]*lian_num

#c初始化装备数目
zhuangbei = sum_ship_list(zhuangbei_S_list, zhuangbei_num_list)
#装备排序
zhuangbei = shell_sort(zhuangbei)
# print("总的装备的排序\n"+str(zhuangbei)+'\n')

for i in range(len(zhuangbei)):
# 最穷的下标
index_min = min_index(lian)
lian[index_min] += zhuangbei[i]

for i in range(len(lian)):
lian[i] += lian_mianji
# print(type+str(lian))

#列表转字典
dict_lian = {}
for i in range(len(lian)):
dict_lian[type+str(i)]=lian[i]
return dict_lian


def recode_zhuangzai(baoguo_k, ship_k, baoguo_v,zhuangzai_Map):
'''记录每一次装载到哪里了'''
if baoguo_k.startswith('一'):
if ship_k not in zhuangzai_Map[0]:
zhuangzai_Map[0][ship_k] = baoguo_v
else:
zhuangzai_Map[0][ship_k] += baoguo_v
elif baoguo_k.startswith('二'):
if ship_k not in zhuangzai_Map[1]:
zhuangzai_Map[1][ship_k] = baoguo_v
else:
zhuangzai_Map[1][ship_k] += baoguo_v
elif baoguo_k.startswith('三'):
if ship_k not in zhuangzai_Map[2]:
zhuangzai_Map[2][ship_k] = baoguo_v
else:
zhuangzai_Map[2][ship_k] += baoguo_v
elif baoguo_k.startswith('四'):
if ship_k not in zhuangzai_Map[3]:
zhuangzai_Map[3][ship_k] = baoguo_v
else:
zhuangzai_Map[3][ship_k] += baoguo_v
elif baoguo_k.startswith('五'):
if ship_k not in zhuangzai_Map[4]:
zhuangzai_Map[4][ship_k] = baoguo_v
else:
zhuangzai_Map[4][ship_k] += baoguo_v
elif baoguo_k.startswith('六'):
if ship_k not in zhuangzai_Map[5]:
zhuangzai_Map[5][ship_k] = baoguo_v
else:
zhuangzai_Map[5][ship_k] += baoguo_v
elif baoguo_k.startswith('七'):
if ship_k not in zhuangzai_Map[6]:
zhuangzai_Map[6][ship_k] = baoguo_v
else:
zhuangzai_Map[6][ship_k] += baoguo_v
elif baoguo_k.startswith('八'):
if ship_k not in zhuangzai_Map[7]:
zhuangzai_Map[7][ship_k] = baoguo_v
else:
zhuangzai_Map[7][ship_k] += baoguo_v
elif baoguo_k.startswith('九'):
if ship_k not in zhuangzai_Map[8]:
zhuangzai_Map[8][ship_k] = baoguo_v
else:
zhuangzai_Map[8][ship_k] += baoguo_v
elif baoguo_k.startswith('十'):
if ship_k not in zhuangzai_Map[9]:
zhuangzai_Map[9][ship_k] = baoguo_v
else:
zhuangzai_Map[9][ship_k] += baoguo_v
elif baoguo_k.startswith('eleven'):
if ship_k not in zhuangzai_Map[10]:
zhuangzai_Map[10][ship_k] = baoguo_v
else:
zhuangzai_Map[10][ship_k] += baoguo_v
elif baoguo_k.startswith('twelve'):
if ship_k not in zhuangzai_Map[11]:
zhuangzai_Map[11][ship_k] = baoguo_v
else:
zhuangzai_Map[11][ship_k] += baoguo_v
return zhuangzai_Map


def zhuangzai_ship(sum_baoguo, sum_ship,zhuangzai_Map):
'''
第一问核心思想
装船的操作,按石头装满装沙子的思想
:param sum_baoguo: 总的包裹
:param sum_ship: 总的船只
:param zhuangzai_Map:装载图
:return: 装载后的船只,装载图
'''
mod_dist = {}
for baoguo_k,baoguo_v in sum_baoguo.items():
flag = False
x=0
for ship_k,ship_v in sum_ship.items():
#如果包裹量小于等于第j艘船的量,装入第j艘船中
x+=1
# if sum_baoguo[i] <= sum_ship[j]:
if baoguo_v <= ship_v:
# sum_ship[j] -= sum_baoguo[i]    #第j量船的载容量减少
flag=True
zhuangzai_Map = recode_zhuangzai(baoguo_k, ship_k , baoguo_v, zhuangzai_Map)    #记录map中的位置
sum_ship[ship_k] -= baoguo_v
# print(baoguo_k+"装到"+ship_k)
break   #开始装第二个包裹
else:
pass    #否则开启第二只船
if len(sum_ship.items())==x and flag==False:
print(baoguo_k+"没有船可装"+str(baoguo_v))
#收集起来
mod_dist[baoguo_k] = baoguo_v
return sum_ship,zhuangzai_Map,mod_dist

def loss_baoguo(sum_baoguo):
sum_ship = {"2万吨级滚装船":38000*0.7}
mod_dist = {}
a = []
b = []
c = []
for baoguo_k, baoguo_v in sum_baoguo.items():
flag = False
x = 0
for ship_k, ship_v in sum_ship.items():
# 如果包裹量小于等于第j艘船的量,装入第j艘船中
x += 1
# if sum_baoguo[i] <= sum_ship[j]:
if baoguo_v <= ship_v:
# sum_ship[j] -= sum_baoguo[i]    #第j量船的载容量减少
flag = True
sum_ship[ship_k] -= baoguo_v
print(baoguo_k+"装到"+ship_k)
if baoguo_k.startswith("二"):
a.append(baoguo_v)
elif baoguo_k.startswith("三"):
b.append(baoguo_v)
else:
c.append(baoguo_v)
break  # 开始装第二个包裹
else:
pass  # 否则开启第二只船
if len(sum_ship.items()) == x and flag == False:
print(baoguo_k + "没有船可装" + str(baoguo_v))
# 收集起来
mod_dist[baoguo_k] = baoguo_v
print(sum(a),sum(b),sum(c))
return sum_ship, mod_dist


if __name__ == '__main__':

'''每个装备对应的面积'''
x_len = np.array([11,10,8,7.5,6.7,6,6.7,7.5,13,9,5.2,5.6,7,8.5])
x_kunan = np.array([3.5,3.2,3.5,3.3,3.3,3,3.2,3.3,12,2.5,3.5,3.5,3.5,2.4])
x_weight = np.array([1.24,1.24,1.24,1.24,1.24,1.24,1,1,1.14,1.1,1.1,1.05,1.24,1.05])
x_S = x_kunan*x_len*x_weight
print("每个装备对应的面积\n"+str(x_S)+'\n')

'''船只融载面积'''
Y = np.array([4000,750,1500,750,750,600,600,1200,300,200,200,200,70,70])
Y = Y*0.75
num_Y = np.array([5,3,9,4,11,10,1,1,31,12,10,200,36,8])
print("船只融载面积\n"+str(shell_sort(Y))+'\n')
# print("所有船总面积sum(Y) = " + str(sum(Y*num_Y)))

''''''''''''''''''''''''''''''''''''''''''''''''''''''''''''''''''''

#一个旅对应的包裹
one_sum_baoguo = zhuang_baoguo([130,100,20,5,8,5,50],[47.74,34.72,27.4164,22.32,21.44,24.75,21.42],54,43.51851852,'一')
two_sum_baoguo = zhuang_baoguo([220,20,5,8,5,50],[34.72,27.4164,22.32,21.44,24.75,21.42],54,43.51851852,'二')
three_sum_baoguo = zhuang_baoguo([120,10,5,8,4,50],[34.72,27.4164,22.32,21.44,24.75,21.42],54,41.6666667,'三')
four_sum_baoguo = zhuang_baoguo([120,50],[27.4164,21.42],35,37.14285714,'四')
five_sum_baoguo = zhuang_baoguo([20,25,30],[21.44,24.75,30.38],35,28.5714285,'五')
six_sum_baoguo_feiji = zhuang_baoguo([40,30],[177.84,21.42],35,28.57142857,'六')#带飞机的旅,先装!!!!!!!
seven_sum_baoguo = zhuang_baoguo([30,15],[34.72,24.75],35,27.57142857,'七')
eight_sum_baoguo = zhuang_baoguo([150],[20.02],35,35.71428571,'八')
night_sum_baoguo = zhuang_baoguo([100],[20.58],35,32.85714286,'九')
ten_sum_baoguo = zhuang_baoguo([100],[21.42],35,28.57142857,'十')
elevn_sum_baoguo = zhuang_baoguo([100,105,18,12,50],[39.68,30.69,27.4164,22.32,21.42],35,71.42857143,'eleven')
twelve_sum_baoguo = zhuang_baoguo([],[],35,42.85714286,'twelve')
# sum_baoguo = [2000,140,1611,41,780,145,17,2510,1,148]   #测试数据1
# sum_baoguo = one_sum_baoguo     #测试装包数据
# print("第一旅的包裹\n"+str(sum_baoguo)+'\n')

# '''这里写几个旅的包裹相加'''
sum_baoguo = Merge(Merge(Merge(Merge(Merge(Merge(Merge(Merge(Merge(Merge(one_sum_baoguo,two_sum_baoguo),three_sum_baoguo),four_sum_baoguo),five_sum_baoguo),seven_sum_baoguo),eight_sum_baoguo),night_sum_baoguo),ten_sum_baoguo),elevn_sum_baoguo),twelve_sum_baoguo)# +.........
# print(sum_baoguo)
first_baoguo = six_sum_baoguo_feiji
# # 总船只
sum_ship = sum_ship_list(Y,num_Y)
dict_ship = ship_list2dict(sum_ship)#船变字典
# print(dict_ship)


# '''包裹与船只分别从大到小排序'''
sum_baoguo = dict(kv_sort(sum_baoguo))
first_baoguo = dict(kv_sort(first_baoguo))
sum_ship = dict(kv_sort(dict_ship))

# zhuangzai_Map = np.array([[0.0]*12]*len(sum_ship))
zhuangzai_Map = [{},{},{},{},{},{},{},{},{},{},{},{}]
sum_zhuangzai_ship,zhuangzai_Map2,mod_dist = zhuangzai_ship(first_baoguo,sum_ship,zhuangzai_Map)
print('总船只装载飞机后的船只装载量\n'+str(sum_zhuangzai_ship)+'\n')
# print("装载图\n"+str(zhuangzai_Map2)+'\n')
# print("总船只\n"+str(sum_ship)+'\n')
# print("总包裹\n"+str(sum_baoguo)+'\n')
print("剩余的包裹\n"+str(mod_dist)+'\n')

sum_zhuangzai_ship,zhuangzai_Map3,mod_dist = zhuangzai_ship(sum_baoguo,sum_zhuangzai_ship,zhuangzai_Map2)
print('总船只装载后的装载量\n'+str(sum_zhuangzai_ship)+'\n')
print("装载图\n"+str(zhuangzai_Map3)+'\n')
print("剩余的包裹\n"+str(mod_dist)+'\n')
sum_ship,mod_dist = loss_baoguo(mod_dist)
print("总船只\n"+str(sum_ship)+'\n')

#输出结果并存储
zhuangzai_Map_df = pd.DataFrame(zhuangzai_Map3)
zhuangzai_Map_df.to_csv('zhuangzai_Map.csv')
\end{lstlisting}
\subsection{问题二沙石遗传与时间轴仿真模型--python源代码}
\begin{lstlisting}[language=python]
import math
import numpy as np
import pandas as pd
import random
import matplotlib.pyplot as plt
from matplotlib import font_manager

my_font = font_manager.FontProperties(fname="C:\Windows\Fonts\msyh.ttc")#微软雅黑字体位置


def append(list,x):
'''numpy数组中插入一个元素在末尾'''
return np.append(list,x)


def max_iii(a):
return a.index(max(a))


def Merge(dict1, dict2):
'''字典合并'''
res = {**dict1, **dict2}
return res


def kv_deng(dic1):
modddd = {}
for k, v in dic1.items():
modddd[k] = v
return modddd


def shell_sort(lists):
'''
按从大到小排序,时间复杂度O(n)
:param lists:传进来的数组,比如船只或者包裹
:return: 返回从大到小的包裹
'''
# 希尔排序
lists = sorted(lists,reverse=True)
return lists


def kv_sort(kv):
'''
键值对由大到小排序
eg:
d = {'d1':2, 'd2':4, 'd4':1,'d3':3,}
res = sorted(d.items(),key=lambda d:d[1],reverse=True)
[('d2', 4), ('d3', 3), ('d1', 2), ('d4', 1)]
:param kv:键值对
:return:元组
'''
res = sorted(kv.items(), key=lambda d: d[1], reverse=True)
return res


def min_index(lists):
'''
返回最穷连的下标
:param lists: 各连的装备面积数目
:return: 最穷的那个连的下标
'''
index_min = 0
for i in range(1, len(lists)):
if lists[i] < lists[index_min]:
index_min = i
return index_min


def jiaopei_1(live1,live2):
'''
遗传交配的第一种方法
:param live1: 父代
:param live2: 父代
:return: 子代
'''
son = [0,0,0,0,0]
for i in range(5):
son[i] = random.randint(min(live1[i],live2[i]),max(live1[i],live2[i]))
return son

def jiaopei_2(live1,live2):
'''
遗传交配的第二种方法
:param live1: 父代
:param live2: 父代
:return: 子代两个
'''
son1 = [0,0,0,0,0]
son2 = [0,0,0,0,0]
x = random.randint(1,4)
for i in range(x):
son1[i] = live1[i]
son2[i] = live2[i]
for i in range(5-x):
son1[4-i] = live2[4-i]
son2[4-i] = live1[4-i]

return son1,son2

def bianyi(geti , gailv):
'''
遗传操作中变异的操作
:param geti 个体
:param gailv 变异概率
:return: 变异后的个体
'''
is_bianyi = False
bianyihou = geti
if gailv > random.random():
# print("变异")
x = random.randint(0, 4)
y = random.randint(0, 4)
if geti[2]==1 and y==2:
geti[2]+=1
is_bianyi=True
elif geti[y] == 0:
geti[y]+=1
is_bianyi=True
else:
if random.random()>0.5:
geti[y]+=1
is_bianyi = True
else:
geti[y]-=1
is_bianyi = True

if geti[2]==1 and x==2:
geti[2]+=1
is_bianyi=True
elif geti[x] == 0:
geti[x]+=1
is_bianyi=True
else:
if random.random()>0.5:
geti[x]+=1
is_bianyi = True
else:
geti[x]-=1
is_bianyi = True

return bianyihou,is_bianyi


def ship_list2dict(sum_ship):
'''
船变字典
:param sum_ship:船的列表
:return: 船的字典
'''
# 船只变成列表
# 列表转字典
dict_ship = {}
for i in range(len(sum_ship)):
if i < 5:
dict_ship["Y1_" + str(i)] = sum_ship[i]
elif i < 8:
dict_ship["Y2_" + str(i - 5)] = sum_ship[i]
elif i < 17:
dict_ship["Y3_" + str(i - 8)] = sum_ship[i]
elif i < 21:
dict_ship["Y4_" + str(i - 17)] = sum_ship[i]
elif i < 32:
dict_ship["Y5_" + str(i - 21)] = sum_ship[i]
elif i < 42:
dict_ship["Y6_" + str(i - 32)] = sum_ship[i]
elif i < 43:
dict_ship["Y7_" + str(i - 42)] = sum_ship[i]
elif i < 44:
dict_ship["Y8_" + str(i - 43)] = sum_ship[i]
elif i < 75:
dict_ship["Y9_" + str(i - 44)] = sum_ship[i]
elif i < 87:
dict_ship["Y10_" + str(i - 75)] = sum_ship[i]
elif i < 97:
dict_ship["Y11_" + str(i - 87)] = sum_ship[i]
elif i < 297:
dict_ship["Y12_" + str(i - 97)] = sum_ship[i]
elif i < 333:
dict_ship["Y13_" + str(i - 297)] = sum_ship[i]
elif i < 341:
dict_ship["Y14_" + str(i - 333)] = sum_ship[i]
return dict_ship


def ship_list2dict_z(sum_ship, z1, z2, z3, z4, z5):
'''
船变字典,带有一个参数的
:param sum_ship:船的列表
:return: 船的字典
'''
# 船只变成列表
# 列表转字典
dict_ship = {}
for i in range(len(sum_ship)):
if i < z1:
dict_ship["Z1_"+str(i)] = sum_ship[i]
elif i < z1+z2:
dict_ship["Z2_"+str(i-z1)] = sum_ship[i]
elif i < z1+z2+z3:
dict_ship["Z3_"+str(i-z1-z2)] = sum_ship[i]
elif i < z1+z2+z3+z4:
dict_ship["Z4_"+str(i-z1-z2-z3)] = sum_ship[i]
elif i < z1+z2+z3+z4+z5:
dict_ship["Z5_"+str(i-z1-z2-z3-z4)] = sum_ship[i]
return dict_ship


def sum_ship_list(ship_S, ship_num):
'''
计算所有的船只数目/装备数目
:param ship_S: 船只面积/装备面积
:param ship_num: 各类船只数目/装备数目
:return: 所有用有船只面积/装备面积
'''
sum_ship = []
if(len(ship_num)==len(ship_S)):
for i in range(len(ship_num)):
for j in range(ship_num[i]):
sum_ship.append(ship_S[i])
else:
print('船数与船面积的两个列表不相等,可能是打错了\n')
return sum_ship


def zhuang_baoguo(zhuangbei_num_list, zhuangbei_S_list, lian_num, lian_mianji, type, lv_num):
'''
给出装备数目,装备面积,要求*均分*给每个连
均分打包思想:谁穷我给谁装备:
装备我从小到大排序;
for 装备的数目:
装备先给连j;
判断哪个连最穷
给穷的连装备

:param zhuangbei_num_list: 各个装备的数目
:param zhuangbei_S_list: 不同装备的面积
:param lian_num: 连的数量
:param lian_mianji: 连中人的面积
:param type:   旅的类型,键
:param lv_num:  旅的数目
:return: 装备打包好了后,下一步就是上船
'''

#初始化连的数目
lian = [0.0]*lian_num

#c初始化装备数目
zhuangbei = sum_ship_list(zhuangbei_S_list, zhuangbei_num_list)
#装备排序
zhuangbei = shell_sort(zhuangbei)
# print("总的装备的排序\n"+str(zhuangbei)+'\n')

for i in range(len(zhuangbei)):
# 最穷的下标
index_min = min_index(lian)
lian[index_min] += zhuangbei[i]

for i in range(len(lian)):
lian[i] += lian_mianji
# print(type+str(lian))

lian2 = []
for i in range(lv_num):
lian2 += lian

#列表转字典
dict_lian = {}
for i in range(len(lian2)):
dict_lian[type+str(i)] = lian2[i]
return dict_lian


def zhuangzai_ship(sum_baoguo, sum_ship):
'''
第一问核心思想
装船的操作,按石头装满装沙子的思想
:param sum_baoguo: 总的包裹
:param sum_ship: 总的船只
:return: 装载后的船只,装载图
'''
mod_dist = {}
for baoguo_k,baoguo_v in sum_baoguo.items():
flag = False
x=0
for ship_k,ship_v in sum_ship.items():
#如果包裹量小于等于第j艘船的量,装入第j艘船中
x+=1
# if sum_baoguo[i] <= sum_ship[j]:
if baoguo_v <= ship_v:
# sum_ship[j] -= sum_baoguo[i]    #第j量船的载容量减少
flag=True
sum_ship[ship_k] -= baoguo_v
# print(baoguo_k+"装到"+ship_k)
break   #开始装第二个包裹
else:
pass    #否则开启第二只船
if len(sum_ship.items()) == x and flag == False:
# print(baoguo_k+"没有船可装"+str(baoguo_v))
#收集起来
mod_dist[baoguo_k] = baoguo_v
return sum_ship,mod_dist


def zhuangzai_zhishengji_ship(sum_baoguo, sum_ship):
'''
装六型旅的操作,按石头装满装沙子的思想
:param sum_baoguo: 总的包裹
:param sum_ship: 总的船只
:return: 装载后的船只,装载图
'''
mod_dist = {}
for baoguo_k,baoguo_v in sum_baoguo.items():
flag = False
x=0
for ship_k,ship_v in sum_ship.items():
#如果包裹量小于等于第j艘船的量,装入第j艘船中
x += 1
# if sum_baoguo[i] <= sum_ship[j]:
if baoguo_v <= ship_v:
if ship_k.startswith("Y1") or ship_k.startswith("Y5"):
# sum_ship[j] -= sum_baoguo[i]    #第j量船的载容量减少
flag=True
sum_ship[ship_k] -= baoguo_v
# print(baoguo_k+"装到"+ship_k)
break   #开始装第二个包裹
else:
pass    #否则开启第二只船
if len(sum_ship.items())==x and flag==False:
# print(baoguo_k+"没有船可装"+str(baoguo_v))
#收集起来
mod_dist[baoguo_k] = baoguo_v
return sum_ship,mod_dist


def zhuangzai_xuanze_ship(sum_baoguo, sum_Z_ship):
'''
剩余包裹装民船
:param sum_baoguo: 总的包裹
:param sum_ship: 总的船只
:return: 装载后的船只,装载图
'''
# 一开始是不适应的
flag = True
for baoguo_k,baoguo_v in sum_baoguo.items():
for ship_k,ship_v in sum_Z_ship.items():
# 白包裹
if baoguo_k.startswith("一") or baoguo_k.startswith("二") or baoguo_k.startswith("三") or baoguo_k.startswith("七") or baoguo_k.startswith("eleven"):
if ship_k.startswith("Z3") or ship_k.startswith("Z4") or ship_k.startswith("Z5"):
#如果白包裹碰到z345,换下一艘船
pass
else:
if baoguo_v <= ship_v:
# 装上船,换下一个包裹
sum_Z_ship[ship_k] -= baoguo_v
sum_baoguo[baoguo_k] = 0
break
#黄包裹
elif baoguo_k.startswith("四") or baoguo_k.startswith("五") or baoguo_k.startswith("六") or baoguo_k.startswith("八") or baoguo_k.startswith("九") or baoguo_k.startswith("十"):
if ship_k.startswith("Z4") or ship_k.startswith("Z5"):
pass
else:
if baoguo_v <= ship_v:
sum_Z_ship[ship_k] -= baoguo_v
sum_baoguo[baoguo_k] = 0
break
#人
else:
if baoguo_v <= ship_v:

sum_Z_ship[ship_k] -= baoguo_v
sum_baoguo[baoguo_k] = 0
break

for baoguo_k, baoguo_v in sum_baoguo.items():
if sum_baoguo[baoguo_k] != 0:
flag=False

return sum_Z_ship,flag


def jizhuangxaing_zhishengji_ship(sum_baoguo, sum_ship):
'''
民船Z3装在直升机
:param sum_baoguo: 总的包裹
:param sum_ship: 总的船只
:return: 装载后的船只,装载图
'''
mod_dist = {}
for baoguo_k,baoguo_v in sum_baoguo.items():
flag = False
x=0
for ship_k,ship_v in sum_ship.items():
#如果包裹量小于等于第j艘船的量,装入第j艘船中
x += 1
# if sum_baoguo[i] <= sum_ship[j]:
if baoguo_v <= ship_v:
if ship_k.startswith("Y1") or ship_k.startswith("Y5"):
# sum_ship[j] -= sum_baoguo[i]    #第j量船的载容量减少
flag=True
sum_ship[ship_k] -= baoguo_v
# print(baoguo_k+"装到"+ship_k)
break   #开始装第二个包裹
else:
pass    #否则开启第二只船
if len(sum_ship.items())==x and flag==False:
# print(baoguo_k+"没有船可装"+str(baoguo_v))
#收集起来
mod_dist[baoguo_k] = baoguo_v
return sum_ship,mod_dist


def zhuangmingchuan(mod_dist, zx):
'''
剩余的包裹装一种类型船的数量
:param mod_dist: 剩余的包裹
:param zx: 船的容积
:return: 船的数目
'''
zx_num = 1
zxy=zx
for baoguo_k, baoguo_v in mod_dist.items():
if zxy>=0 and zxy>=baoguo_v:
zxy-=baoguo_v
else:
zxy=zx
zx_num+=1
return zx_num


def surt_time(live):
'''
计算适应度的函数,适应度为时间
:param live: 存活个体[z1,z2...,z5]
:return: 适应时间
'''
num_matou = [10, 5, 1, 3, 2, 10, 1, 4, 3, 6, 5, 3, 5, 3, 3, 7]
w_mat = [30000, 20000, 1000, 30000, 1000, 1000, 1000, 30000, 1000, 30000, 1000, 1000, 500, 1000, 30000, 1000]
t_matou = [0] * 71
W_matou = sum_ship_list(w_mat,num_matou)
numZ = np.array(live)
# 第一行向量,船的数目
num_ship = np.hstack((num_Y, numZ))
# print("输入船的数量"+str(num_ship))
# print(num_ship)
# 第二行向量,船的吨位
w_ship = [18500, 4170, 4800, 4170, 4800, 2000, 1650, 1800, 850, 850, 800, 600, 128, 85, 20000, 30000, 30000, 10000,
5000]
# 第三行向量,船的时间
t_ship = [5, 5, 5, 5, 5, 4, 4, 4, 4, 3, 3, 3, 1, 1, 10, 15, 18, 5, 4]

'''适应度的计算:时间t0'''
t0 = -1
while True:
t0+=1

if len(num_ship) != len(w_ship) or len(num_ship) != len(t_ship) or len(W_matou) != len(t_matou):
print("输入错误,请查看船或码头个数")
break

#检索71次
for i in range(len(t_matou)):
#如果有码头是空闲的
if t_matou[i]==t0:
#检索对应的船
manzu_w_ship = []
manzu_j_ship = []
for j in range(len(w_ship)):
#如果满足条件
if W_matou[i]>w_ship[j]:
manzu_w_ship.append(w_ship[j])#记录重量和对应重量的位置
manzu_j_ship.append(j)
#取出最大满足吨位的船的下标
j_max = max_iii(manzu_w_ship)
if num_ship[j_max]>0:
#船的数量减一
num_ship[j_max] -= 1
#码头对应时间加上tk
# print(t_matou)
t_matou[i] += t_ship[j_max]
if num_ship[j_max]==0:
w_ship[j_max]=0
# print(t_matou)
#跳出循环
# print(num_ship)

if t0>max(t_matou) and sum(num_ship)==0:
break
# print(t0)
return t0


def is_surt_manzai(Z_X,mod_dist):
'''
判断某一个个体是否满足能装载
:return: 是否能装满
'''
Zx_w = [19000 * 0.7, 26000 * 0.7, 38000 * 0.7, 6000 * 0.7, 2000 * 0.7]
sum_Z_ship = sum_ship_list(Zx_w, Z_X)
dict_ship_Z = ship_list2dict_z(sum_Z_ship, Z_X[0], Z_X[1], Z_X[2], Z_X[3], Z_X[4])  # 船变字典
dict_ship_Z['Z3_' + str(Z_X[2])] = 2494.836677158
sum_Z_ship_dict = dict(kv_sort(dict_ship_Z))  # 排序
moddd = kv_deng(mod_dist)
shiying_ship, is_sure = zhuangzai_xuanze_ship(moddd, sum_Z_ship_dict)
# print("是否适应\n" + str(is_sure) + '\n')

return is_sure


if __name__ == '__main__':

'''每个装备对应的面积'''
x_len = np.array([11,10,8,7.5,6.7,6,6.7,7.5,13,9,5.2,5.6,7,8.5])
x_kunan = np.array([3.5,3.2,3.5,3.3,3.3,3,3.2,3.3,12,2.5,3.5,3.5,3.5,2.4])
x_weight = np.array([1.24,1.24,1.24,1.24,1.24,1.24,1,1,1.14,1.1,1.1,1.05,1.24,1.05])
x_S = x_kunan*x_len*x_weight
# print("每个装备对应的面积\n"+str(x_S)+'\n')

'''船只融载面积'''
Y = np.array([4000,750,1500,750,750,600,600,1200,300,200,200,200,70,70])
Y = Y*0.75
num_Y = np.array([5,3,9,4,11,10,1,1,31,12,10,200,36,8])
# print("船只融载面积\n"+str(shell_sort(Y))+'\n')
# print("所有船总面积sum(Y) = " + str(sum(Y*num_Y)))

''''''''''''''''''''''''''''''''''''''''''''''''''''''''''''''''''''

#一个旅对应的包裹
one_sum_baoguo = zhuang_baoguo([130,100,20,5,8,5,50],[47.74,34.72,27.4164,22.32,21.44,24.75,21.42],54,43.51851852,'一',12)
two_sum_baoguo = zhuang_baoguo([220,20,5,8,5,50],[34.72,27.4164,22.32,21.44,24.75,21.42],54,43.51851852,'二',3)
three_sum_baoguo = zhuang_baoguo([120,10,5,8,4,50],[34.72,27.4164,22.32,21.44,24.75,21.42],54,41.6666667,'三',3)
four_sum_baoguo = zhuang_baoguo([120,50],[27.4164,21.42],35,37.14285714,'四',3)
five_sum_baoguo = zhuang_baoguo([20,25,30],[21.44,24.75,30.38],35,28.5714285,'五',3)
six_sum_baoguo_feiji = zhuang_baoguo([40,30],[177.84,21.42],35,28.57142857,'六',5)#带飞机的旅,先装!!!!!!!
seven_sum_baoguo = zhuang_baoguo([30,15],[34.72,24.75],35,27.57142857,'七',5)
eight_sum_baoguo = zhuang_baoguo([150],[20.02],35,35.71428571,'八',3)
night_sum_baoguo = zhuang_baoguo([100],[20.58],35,32.85714286,'九',3)
ten_sum_baoguo = zhuang_baoguo([100],[21.42],35,28.57142857,'十',3)
elevn_sum_baoguo = zhuang_baoguo([100,105,18,12,50],[39.68,30.69,27.4164,22.32,21.42],35,71.42857143,'eleven',6)
twelve_sum_baoguo = zhuang_baoguo([],[],35,42.85714286,'twelve',10)
# sum_baoguo = [2000,140,1611,41,780,145,17,2510,1,148]   #测试数据1
# sum_baoguo = one_sum_baoguo     #测试装包数据
# print("第一旅的包裹\n"+str(sum_baoguo)+'\n')

# '''这里写几个旅的包裹相加'''
sum_baoguo = Merge(Merge(Merge(Merge(Merge(Merge(Merge(Merge(Merge(Merge(one_sum_baoguo,two_sum_baoguo),three_sum_baoguo),four_sum_baoguo),five_sum_baoguo),seven_sum_baoguo),eight_sum_baoguo),night_sum_baoguo),ten_sum_baoguo),elevn_sum_baoguo),twelve_sum_baoguo)# +.........
# print(sum_baoguo)
first_baoguo = six_sum_baoguo_feiji
# # 总船只
sum_ship = sum_ship_list(Y,num_Y)
dict_ship = ship_list2dict(sum_ship)#船变字典
print(dict_ship)


'''包裹与船只分别从大到小排序'''
sum_baoguo = dict(kv_sort(sum_baoguo))
first_baoguo = dict(kv_sort(first_baoguo))
sum_ship = dict(kv_sort(dict_ship))
# print("总船只\n"+str(sum_ship)+'\n')

# '''装直升机的操作'''
sum_zhuangzai_ship, mod_dist = zhuangzai_zhishengji_ship(first_baoguo, sum_ship)
print('总船只装载飞机后的船只装载量\n'+str(sum_zhuangzai_ship)+'\n')
# print("总船只\n"+str(sum_ship)+'\n')
print("总包裹\n"+str(sum_baoguo)+'\n')
# '''106个直升机的包裹
#  总面积24105.163322842
#   开一个Z3装得下13300*2
#   剩下2494.836677158
#   '''
# print("剩余的包裹\n"+str(mod_dist)+'\n')
# z3 = 1
#
# '''装人的操作'''
sum_zhuangzai_ship,mod_dist = zhuangzai_ship(sum_baoguo,sum_zhuangzai_ship)
print('总船只装载后的装载量\n'+str(sum_zhuangzai_ship)+'\n')
print("装民船的面积的面积\n"+str(sum(mod_dist.values())))
mod_dist = dict(kv_sort(mod_dist))
# print("剩余的包裹\n"+str(mod_dist)+'\n')
#
#
# z1 = zhuangmingchuan(mod_dist,19000*0.7)
# z2 = zhuangmingchuan(mod_dist,26000*0.7)
# z3 += zhuangmingchuan(mod_dist,38000*0.7)
#Z的最大上限
# print(z1,z2,z3) #[25, 18, 14, 4, 11]

#############################################################################################################################
#存活个体数
w = 50
live = []
Zx_w = [19000 * 0.7, 26000 * 0.7, 38000 * 0.7, 6000 * 0.7, 2000 * 0.7]
Zx_num = [25, 18, 6, 4, 11]

#初始化适应个体
while len(live) != w:
# 总民用船的数目
random_W = np.array([random.random(),random.random(),random.random(),random.random(),random.random()])
# 初始个体数目
num_Zx = np.trunc(random_W*Zx_num).astype(np.int8)#取整

# # 总民用船船只
sum_Z_ship = sum_ship_list(Zx_w, num_Zx)
# print("总民用船船只\n" + str(sum_Z_ship) + '\n')
dict_ship_Z = ship_list2dict_z(sum_Z_ship,num_Zx[0],num_Zx[1],num_Zx[2],num_Zx[3],num_Zx[4])# 船变字典
dict_ship_Z['Z3_'+str(num_Zx[2])] = 2494.836677158
sum_Z_ship_dict = dict(kv_sort(dict_ship_Z))# 排序
# print("民用船船只字典和个数\n" + str(sum_Z_ship_dict) + '\n')
# print("剩余的包裹\n" + str(mod_dist) + '\n')
# print(mod_dist)
moddd = kv_deng(mod_dist)

shiying_ship, is_sure = zhuangzai_xuanze_ship(moddd, sum_Z_ship_dict)
# print("是否适应\n" + str(is_sure) + '\n')
moddd = kv_deng(mod_dist)

if is_sure:
num_Zx[2] += 1
live.append(num_Zx.tolist())
# print("装完物体后民用船适应度\n" + str(shiying_ship) + '\n')

# break
# print("适应个体:"+str(live)+'\n')

#############################################################################################################################
# 计算适应度
# t0 = surt_time(live[0])
# print("计算适应度::"+str(t0))
# 测试交配
# print(live[0],live[1])
# print(jiaopei_1(live[0],live[1]))
# print(jiaopei_2(live[0],live[1]))
#
# #测试变异
# print(live[0])
# print(bianyi(live[0],0.8))
#
# 测试是否满足适应
# print(is_surt_manzai(live[4],mod_dist))
#
#############################################################################################################################
# 遗传的实现
min_t = []
duiying_num = []
for i in range(100):
the_all_son = live
bianyilv = 0.2
for i in range(w):
#变异
bianyi_son,is_bianyi = bianyi(live[i],bianyilv)
if is_bianyi:
the_all_son.append(bianyi_son)
# print(len(the_all_son))

#交配
random.shuffle(live)
for i in range(0,w-1,2):
son = jiaopei_1(live[i],live[i+1])
the_all_son.append(son)
son1,son2 = jiaopei_2(live[i],live[i+1])
the_all_son.append(son1)
the_all_son.append(son2)
# the_all_son.append()
# print(len(the_all_son))


t0_all = []
surt_son = []
for i in range(len(the_all_son)):
if is_surt_manzai(the_all_son[i],mod_dist):
t0_all.append(surt_time(the_all_son[i]))
surt_son.append(the_all_son[i])

chongfu = []
for i in range(len(surt_son)):
for j in range(i,len(surt_son)):
if i!=j and surt_son[i]==surt_son[j]:
# print(j)
chongfu.append(i)
# print(t0_all)
# print(surt_son)

for i in range(len(chongfu),0,-1):
surt_son.pop(i)
t0_all.pop(i)

# print("变异后个体:"+str(len(the_all_son)))
num_ship = []
index = []
for i in range(len(surt_son)):
# if t0_all[i] == 21:
#     print(surt_son[i])
num_ship.append(sum(surt_son[i]))
index.append(i)
# print(t0_all)
# print(num_ship)

x = np.vstack((np.array(t0_all),np.array(num_ship),np.array(index)))
# 第一行排序
x = x.T[np.lexsort(x[::-1,:])].T
min_t.append(x[0][0])
duiying_num.append(x[1][0])

youxiu_geti = []
for i in range(w):
youxiu_geti.append(surt_son[x[2][i]])
print(youxiu_geti)
print("时间:"+str(x[0]))
print("船数:"+str(x[1]))
live = youxiu_geti

plt.plot(duiying_num, label='种群中对应船只数量', )

plt.plot(min_t, label='种群中最小装载时间', color="r")
plt.xlabel("迭代次数", fontproperties=my_font)
plt.title("适应度变化趋势", fontproperties=my_font)
plt.legend(prop=my_font, loc=0)
plt.grid(alpha=0.3, linestyle="--")  # alpha为透明度 0-1

plt.show()
#############################################################################################################################
######遍历Zx_num = [25, 18, 14, 4, 11]
# file = 'test.txt'
# with open(file, 'a+') as f:
#     for i_a in range(25):
#         for i_b in range(18):
#             for i_c in range(6):
#                 for i_d in range(4):
#                     for i_e in range(11):
#                         # print(i_a,i_b,i_c,i_d,i_e)
#                         if is_surt_manzai([i_a,i_b,i_c,i_d,i_e], mod_dist):
#                             tqqq = surt_time([i_a,i_b,i_c,i_d,i_e])
#                             f.write(str(tqqq)+","+str([i_a,i_b,i_c,i_d,i_e])+","+str(sum([i_a,i_b,i_c,i_d,i_e]))+"\n")
#                             print("*"*20+str(tqqq))
#############################################################################################################################


\end{lstlisting}
\subsection{问题三改进时间轴仿真模型--python源代码}
\begin{lstlisting}[language=python]
import math
import numpy as np
import pandas as pd
import random
import matplotlib.pyplot as plt
from matplotlib import font_manager

my_font = font_manager.FontProperties(fname="C:\Windows\Fonts\msyh.ttc")#微软雅黑字体位置


def append(list,x):
'''numpy数组中插入一个元素在末尾'''
return np.append(list,x)


def max_iii(a):
return a.index(max(a))


def Merge(dict1, dict2):
'''字典合并'''
res = {**dict1, **dict2}
return res


def kv_deng(dic1):
modddd = {}
for k, v in dic1.items():
modddd[k] = v
return modddd


def shell_sort(lists):
'''
按从大到小排序,时间复杂度O(n)
:param lists:传进来的数组,比如船只或者包裹
:return: 返回从大到小的包裹
'''
# 希尔排序
lists = sorted(lists,reverse=True)
return lists


def kv_sort(kv):
'''
键值对由大到小排序
eg:
d = {'d1':2, 'd2':4, 'd4':1,'d3':3,}
res = sorted(d.items(),key=lambda d:d[1],reverse=True)
[('d2', 4), ('d3', 3), ('d1', 2), ('d4', 1)]
:param kv:键值对
:return:元组
'''
res = sorted(kv.items(), key=lambda d: d[1], reverse=True)
return res


def min_index(lists):
'''
返回最穷连的下标
:param lists: 各连的装备面积数目
:return: 最穷的那个连的下标
'''
index_min = 0
for i in range(1, len(lists)):
if lists[i] < lists[index_min]:
index_min = i
return index_min


def jiaopei_1(live1,live2):
'''
遗传交配的第一种方法
:param live1: 父代
:param live2: 父代
:return: 子代
'''
son = [0,0,0,0,0]
for i in range(5):
son[i] = random.randint(min(live1[i],live2[i]),max(live1[i],live2[i]))
return son

def jiaopei_2(live1,live2):
'''
遗传交配的第二种方法
:param live1: 父代
:param live2: 父代
:return: 子代两个
'''
son1 = [0,0,0,0,0]
son2 = [0,0,0,0,0]
x = random.randint(1,4)
for i in range(x):
son1[i] = live1[i]
son2[i] = live2[i]
for i in range(5-x):
son1[4-i] = live2[4-i]
son2[4-i] = live1[4-i]

return son1,son2

def bianyi(geti , gailv):
'''
遗传操作中变异的操作
:param geti 个体
:param gailv 变异概率
:return: 变异后的个体
'''
is_bianyi = False
bianyihou = geti
if gailv > random.random():
# print("变异")
x = random.randint(0, 4)
y = random.randint(0, 4)
if geti[2]==1 and y==2:
geti[2]+=1
is_bianyi=True
elif geti[y] == 0:
geti[y]+=1
is_bianyi=True
else:
if random.random()>0.5:
geti[y]+=1
is_bianyi = True
else:
geti[y]-=1
is_bianyi = True

if geti[2]==1 and x==2:
geti[2]+=1
is_bianyi=True
elif geti[x] == 0:
geti[x]+=1
is_bianyi=True
else:
if random.random()>0.5:
geti[x]+=1
is_bianyi = True
else:
geti[x]-=1
is_bianyi = True

return bianyihou,is_bianyi


def ship_list2dict(sum_ship):
'''
船变字典
:param sum_ship:船的列表
:return: 船的字典
'''
# 船只变成列表
# 列表转字典
dict_ship = {}
for i in range(len(sum_ship)):
if i < 5:
dict_ship["Y1_" + str(i)] = sum_ship[i]
elif i < 8:
dict_ship["Y2_" + str(i - 5)] = sum_ship[i]
elif i < 17:
dict_ship["Y3_" + str(i - 8)] = sum_ship[i]
elif i < 21:
dict_ship["Y4_" + str(i - 17)] = sum_ship[i]
elif i < 32:
dict_ship["Y5_" + str(i - 21)] = sum_ship[i]
elif i < 42:
dict_ship["Y6_" + str(i - 32)] = sum_ship[i]
elif i < 43:
dict_ship["Y7_" + str(i - 42)] = sum_ship[i]
elif i < 44:
dict_ship["Y8_" + str(i - 43)] = sum_ship[i]
elif i < 75:
dict_ship["Y9_" + str(i - 44)] = sum_ship[i]
elif i < 87:
dict_ship["Y10_" + str(i - 75)] = sum_ship[i]
elif i < 97:
dict_ship["Y11_" + str(i - 87)] = sum_ship[i]
elif i < 297:
dict_ship["Y12_" + str(i - 97)] = sum_ship[i]
elif i < 333:
dict_ship["Y13_" + str(i - 297)] = sum_ship[i]
elif i < 341:
dict_ship["Y14_" + str(i - 333)] = sum_ship[i]
return dict_ship


def ship_list2dict_z(sum_ship, z1, z2, z3, z4, z5):
'''
船变字典,带有一个参数的
:param sum_ship:船的列表
:return: 船的字典
'''
# 船只变成列表
# 列表转字典
dict_ship = {}
for i in range(len(sum_ship)):
if i < z1:
dict_ship["Z1_"+str(i)] = sum_ship[i]
elif i < z1+z2:
dict_ship["Z2_"+str(i-z1)] = sum_ship[i]
elif i < z1+z2+z3:
dict_ship["Z3_"+str(i-z1-z2)] = sum_ship[i]
elif i < z1+z2+z3+z4:
dict_ship["Z4_"+str(i-z1-z2-z3)] = sum_ship[i]
elif i < z1+z2+z3+z4+z5:
dict_ship["Z5_"+str(i-z1-z2-z3-z4)] = sum_ship[i]
return dict_ship


def sum_ship_list(ship_S, ship_num):
'''
计算所有的船只数目/装备数目
:param ship_S: 船只面积/装备面积
:param ship_num: 各类船只数目/装备数目
:return: 所有用有船只面积/装备面积
'''
sum_ship = []
if(len(ship_num)==len(ship_S)):
for i in range(len(ship_num)):
for j in range(ship_num[i]):
sum_ship.append(ship_S[i])
else:
print('船数与船面积的两个列表不相等,可能是打错了\n')
return sum_ship


def zhuang_baoguo(zhuangbei_num_list, zhuangbei_S_list, lian_num, lian_mianji, type, lv_num):
'''
给出装备数目,装备面积,要求*均分*给每个连
均分打包思想:谁穷我给谁装备:
装备我从小到大排序;
for 装备的数目:
装备先给连j;
判断哪个连最穷
给穷的连装备

:param zhuangbei_num_list: 各个装备的数目
:param zhuangbei_S_list: 不同装备的面积
:param lian_num: 连的数量
:param lian_mianji: 连中人的面积
:param type:   旅的类型,键
:param lv_num:  旅的数目
:return: 装备打包好了后,下一步就是上船
'''

#初始化连的数目
lian = [0.0]*lian_num

#c初始化装备数目
zhuangbei = sum_ship_list(zhuangbei_S_list, zhuangbei_num_list)
#装备排序
zhuangbei = shell_sort(zhuangbei)
# print("总的装备的排序\n"+str(zhuangbei)+'\n')

for i in range(len(zhuangbei)):
# 最穷的下标
index_min = min_index(lian)
lian[index_min] += zhuangbei[i]

for i in range(len(lian)):
lian[i] += lian_mianji
# print(type+str(lian))

lian2 = []
for i in range(lv_num):
lian2 += lian

#列表转字典
dict_lian = {}
for i in range(len(lian2)):
dict_lian[type+str(i)] = lian2[i]
return dict_lian


def zhuangzai_ship(sum_baoguo, sum_ship):
'''
第一问核心思想
装船的操作,按石头装满装沙子的思想
:param sum_baoguo: 总的包裹
:param sum_ship: 总的船只
:return: 装载后的船只,装载图
'''
mod_dist = {}
for baoguo_k,baoguo_v in sum_baoguo.items():
flag = False
x=0
for ship_k,ship_v in sum_ship.items():
#如果包裹量小于等于第j艘船的量,装入第j艘船中
x+=1
# if sum_baoguo[i] <= sum_ship[j]:
if baoguo_v <= ship_v:
# sum_ship[j] -= sum_baoguo[i]    #第j量船的载容量减少
flag=True
sum_ship[ship_k] -= baoguo_v
# print(baoguo_k+"装到"+ship_k)
break   #开始装第二个包裹
else:
pass    #否则开启第二只船
if len(sum_ship.items()) == x and flag == False:
# print(baoguo_k+"没有船可装"+str(baoguo_v))
#收集起来
mod_dist[baoguo_k] = baoguo_v
return sum_ship,mod_dist


def zhuangzai_zhishengji_ship(sum_baoguo, sum_ship):
'''
装六型旅的操作,按石头装满装沙子的思想
:param sum_baoguo: 总的包裹
:param sum_ship: 总的船只
:return: 装载后的船只,装载图
'''
mod_dist = {}
for baoguo_k,baoguo_v in sum_baoguo.items():
flag = False
x=0
for ship_k,ship_v in sum_ship.items():
#如果包裹量小于等于第j艘船的量,装入第j艘船中
x += 1
# if sum_baoguo[i] <= sum_ship[j]:
if baoguo_v <= ship_v:
if ship_k.startswith("Y1") or ship_k.startswith("Y5"):
# sum_ship[j] -= sum_baoguo[i]    #第j量船的载容量减少
flag=True
sum_ship[ship_k] -= baoguo_v
# print(baoguo_k+"装到"+ship_k)
break   #开始装第二个包裹
else:
pass    #否则开启第二只船
if len(sum_ship.items())==x and flag==False:
# print(baoguo_k+"没有船可装"+str(baoguo_v))
#收集起来
mod_dist[baoguo_k] = baoguo_v
return sum_ship,mod_dist


def zhuangzai_xuanze_ship(sum_baoguo, sum_Z_ship):
'''
剩余包裹装民船
:param sum_baoguo: 总的包裹
:param sum_ship: 总的船只
:return: 装载后的船只,装载图
'''
# 一开始是不适应的
flag = True
for baoguo_k,baoguo_v in sum_baoguo.items():
for ship_k,ship_v in sum_Z_ship.items():
# 白包裹
if baoguo_k.startswith("一") or baoguo_k.startswith("二") or baoguo_k.startswith("三") or baoguo_k.startswith("七") or baoguo_k.startswith("eleven"):
if ship_k.startswith("Z3") or ship_k.startswith("Z4") or ship_k.startswith("Z5"):
#如果白包裹碰到z345,换下一艘船
pass
else:
if baoguo_v <= ship_v:
# 装上船,换下一个包裹
sum_Z_ship[ship_k] -= baoguo_v
sum_baoguo[baoguo_k] = 0
break
#黄包裹
elif baoguo_k.startswith("四") or baoguo_k.startswith("五") or baoguo_k.startswith("六") or baoguo_k.startswith("八") or baoguo_k.startswith("九") or baoguo_k.startswith("十"):
if ship_k.startswith("Z4") or ship_k.startswith("Z5"):
pass
else:
if baoguo_v <= ship_v:
sum_Z_ship[ship_k] -= baoguo_v
sum_baoguo[baoguo_k] = 0
break
#人
else:
if baoguo_v <= ship_v:

sum_Z_ship[ship_k] -= baoguo_v
sum_baoguo[baoguo_k] = 0
break

for baoguo_k, baoguo_v in sum_baoguo.items():
if sum_baoguo[baoguo_k] != 0:
flag=False

return sum_Z_ship,flag


def jizhuangxaing_zhishengji_ship(sum_baoguo, sum_ship):
'''
民船Z3装在直升机
:param sum_baoguo: 总的包裹
:param sum_ship: 总的船只
:return: 装载后的船只,装载图
'''
mod_dist = {}
for baoguo_k,baoguo_v in sum_baoguo.items():
flag = False
x=0
for ship_k,ship_v in sum_ship.items():
#如果包裹量小于等于第j艘船的量,装入第j艘船中
x += 1
# if sum_baoguo[i] <= sum_ship[j]:
if baoguo_v <= ship_v:
if ship_k.startswith("Y1") or ship_k.startswith("Y5"):
# sum_ship[j] -= sum_baoguo[i]    #第j量船的载容量减少
flag=True
sum_ship[ship_k] -= baoguo_v
# print(baoguo_k+"装到"+ship_k)
break   #开始装第二个包裹
else:
pass    #否则开启第二只船
if len(sum_ship.items())==x and flag==False:
# print(baoguo_k+"没有船可装"+str(baoguo_v))
#收集起来
mod_dist[baoguo_k] = baoguo_v
return sum_ship,mod_dist


def zhuangmingchuan(mod_dist, zx):
'''
剩余的包裹装一种类型船的数量
:param mod_dist: 剩余的包裹
:param zx: 船的容积
:return: 船的数目
'''
zx_num = 1
zxy=zx
for baoguo_k, baoguo_v in mod_dist.items():
if zxy>=0 and zxy>=baoguo_v:
zxy-=baoguo_v
else:
zxy=zx
zx_num+=1
return zx_num


def surt_time(live):
'''
计算适应度的函数,适应度为时间
:param live: 存活个体[z1,z2...,z5]
:return: 适应时间
'''
num_matou = [10, 5, 1, 3, 2, 10, 1, 4, 3, 6, 5, 3, 5, 3, 3, 7]
w_mat = [30000, 20000, 1000, 30000, 1000, 1000, 1000, 30000, 1000, 30000, 1000, 1000, 500, 1000, 30000, 1000]
t_matou = [0] * 71
W_matou = sum_ship_list(w_mat,num_matou)
numZ = np.array(live)
# 第一行向量,船的数目
num_ship = np.hstack((num_Y, numZ))
# print("输入船的数量"+str(num_ship))
# print(num_ship)
# 第二行向量,船的吨位
w_ship = [18500, 4170, 4800, 4170, 4800, 2000, 1650, 1800, 850, 850, 800, 600, 128, 85, 20000, 30000, 30000, 10000,
5000]
# 第三行向量,船的时间
t_ship = [5, 5, 5, 5, 5, 4, 4, 4, 4, 3, 3, 3, 1, 1, 10, 15, 18, 5, 4]

'''适应度的计算:时间t0'''
# 销毁时间
rtime = 18
t0 = -1
while True:
t0+=1

if len(num_ship) != len(w_ship) or len(num_ship) != len(t_ship) or len(W_matou) != len(t_matou):
print("输入错误,请查看船或码头个数")
break

if t0 == rtime:
# print("码头被摧毁!!!")
pass

#检索71次
for i in range(len(t_matou)):
if t0 == rtime:
#摧毁对应码头
t_matou[0]=-1
t_matou[1]=-1
t_matou[10]=-1
t_matou[40]=-1
t_matou[46]=-1
t_matou[47]=-1
pass
#如果有码头是空闲的
if t_matou[i]==t0:
#检索对应的船
manzu_w_ship = []
manzu_j_ship = []
for j in range(len(w_ship)):
#如果满足条件
if W_matou[i]>w_ship[j]:
manzu_w_ship.append(w_ship[j])#记录重量和对应重量的位置
manzu_j_ship.append(j)
#取出最大满足吨位的船的下标
j_max = max_iii(manzu_w_ship)
if num_ship[j_max]>0:
#船的数量减一
num_ship[j_max] -= 1
#码头对应时间加上tk
# print(t_matou)
t_matou[i] += t_ship[j_max]
if i==0 or i==1 or i==40 or i==46 or i== 47 or i== 10:
# print(str(j_max) + "装载到" + str(i) + "码头里")
if t_matou[i]>rtime:
num_ship[j_max] += 1
if num_ship[j_max]==0:
w_ship[j_max]=0
# print(t_matou)
#跳出循环
# print(num_ship)

if t0>max(t_matou) and sum(num_ship)==0:
break
# print(t0)
return t0


def is_surt_manzai(Z_X,mod_dist):
'''
判断某一个个体是否满足能装载
:return: 是否能装满
'''
Zx_w = [19000 * 0.7, 26000 * 0.7, 38000 * 0.7, 6000 * 0.7, 2000 * 0.7]
sum_Z_ship = sum_ship_list(Zx_w, Z_X)
dict_ship_Z = ship_list2dict_z(sum_Z_ship, Z_X[0], Z_X[1], Z_X[2], Z_X[3], Z_X[4])  # 船变字典
dict_ship_Z['Z3_' + str(Z_X[2])] = 2494.836677158
sum_Z_ship_dict = dict(kv_sort(dict_ship_Z))  # 排序
moddd = kv_deng(mod_dist)
shiying_ship, is_sure = zhuangzai_xuanze_ship(moddd, sum_Z_ship_dict)
# print("是否适应\n" + str(is_sure) + '\n')

return is_sure


if __name__ == '__main__':

'''每个装备对应的面积'''
x_len = np.array([11,10,8,7.5,6.7,6,6.7,7.5,13,9,5.2,5.6,7,8.5])
x_kunan = np.array([3.5,3.2,3.5,3.3,3.3,3,3.2,3.3,12,2.5,3.5,3.5,3.5,2.4])
x_weight = np.array([1.24,1.24,1.24,1.24,1.24,1.24,1,1,1.14,1.1,1.1,1.05,1.24,1.05])
x_S = x_kunan*x_len*x_weight
# print("每个装备对应的面积\n"+str(x_S)+'\n')

'''船只融载面积'''
Y = np.array([4000,750,1500,750,750,600,600,1200,300,200,200,200,70,70])
Y = Y*0.75
num_Y = np.array([5,3,9,4,11,10,1,1,31,12,10,200,36,8])
# print("船只融载面积\n"+str(shell_sort(Y))+'\n')
# print("所有船总面积sum(Y) = " + str(sum(Y*num_Y)))

''''''''''''''''''''''''''''''''''''''''''''''''''''''''''''''''''''

#一个旅对应的包裹
one_sum_baoguo = zhuang_baoguo([130,100,20,5,8,5,50],[47.74,34.72,27.4164,22.32,21.44,24.75,21.42],54,43.51851852,'一',12)
two_sum_baoguo = zhuang_baoguo([220,20,5,8,5,50],[34.72,27.4164,22.32,21.44,24.75,21.42],54,43.51851852,'二',3)
three_sum_baoguo = zhuang_baoguo([120,10,5,8,4,50],[34.72,27.4164,22.32,21.44,24.75,21.42],54,41.6666667,'三',3)
four_sum_baoguo = zhuang_baoguo([120,50],[27.4164,21.42],35,37.14285714,'四',3)
five_sum_baoguo = zhuang_baoguo([20,25,30],[21.44,24.75,30.38],35,28.5714285,'五',3)
six_sum_baoguo_feiji = zhuang_baoguo([40,30],[177.84,21.42],35,28.57142857,'六',5)#带飞机的旅,先装!!!!!!!
seven_sum_baoguo = zhuang_baoguo([30,15],[34.72,24.75],35,27.57142857,'七',5)
eight_sum_baoguo = zhuang_baoguo([150],[20.02],35,35.71428571,'八',3)
night_sum_baoguo = zhuang_baoguo([100],[20.58],35,32.85714286,'九',3)
ten_sum_baoguo = zhuang_baoguo([100],[21.42],35,28.57142857,'十',3)
elevn_sum_baoguo = zhuang_baoguo([100,105,18,12,50],[39.68,30.69,27.4164,22.32,21.42],35,71.42857143,'eleven',6)
twelve_sum_baoguo = zhuang_baoguo([],[],35,42.85714286,'twelve',10)
# sum_baoguo = [2000,140,1611,41,780,145,17,2510,1,148]   #测试数据1
# sum_baoguo = one_sum_baoguo     #测试装包数据
# print("第一旅的包裹\n"+str(sum_baoguo)+'\n')

# '''这里写几个旅的包裹相加'''
sum_baoguo = Merge(Merge(Merge(Merge(Merge(Merge(Merge(Merge(Merge(Merge(one_sum_baoguo,two_sum_baoguo),three_sum_baoguo),four_sum_baoguo),five_sum_baoguo),seven_sum_baoguo),eight_sum_baoguo),night_sum_baoguo),ten_sum_baoguo),elevn_sum_baoguo),twelve_sum_baoguo)# +.........
# print(sum_baoguo)
first_baoguo = six_sum_baoguo_feiji
# # 总船只
sum_ship = sum_ship_list(Y,num_Y)
dict_ship = ship_list2dict(sum_ship)#船变字典
print(dict_ship)


'''包裹与船只分别从大到小排序'''
sum_baoguo = dict(kv_sort(sum_baoguo))
first_baoguo = dict(kv_sort(first_baoguo))
sum_ship = dict(kv_sort(dict_ship))
# print("总船只\n"+str(sum_ship)+'\n')

# '''装直升机的操作'''
sum_zhuangzai_ship, mod_dist = zhuangzai_zhishengji_ship(first_baoguo, sum_ship)
print('总船只装载飞机后的船只装载量\n'+str(sum_zhuangzai_ship)+'\n')
# print("总船只\n"+str(sum_ship)+'\n')
print("总包裹\n"+str(sum_baoguo)+'\n')
# '''106个直升机的包裹
#  总面积24105.163322842
#   开一个Z3装得下13300*2
#   剩下2494.836677158
#   '''
# print("剩余的包裹\n"+str(mod_dist)+'\n')
# z3 = 1
#
# '''装人的操作'''
sum_zhuangzai_ship,mod_dist = zhuangzai_ship(sum_baoguo,sum_zhuangzai_ship)
print('总船只装载后的装载量\n'+str(sum_zhuangzai_ship)+'\n')
print("装民船的面积的面积\n"+str(sum(mod_dist.values())))
mod_dist = dict(kv_sort(mod_dist))
# print("剩余的包裹\n"+str(mod_dist)+'\n')
#
#
# z1 = zhuangmingchuan(mod_dist,19000*0.7)
# z2 = zhuangmingchuan(mod_dist,26000*0.7)
# z3 += zhuangmingchuan(mod_dist,38000*0.7)
#Z的最大上限
# print(z1,z2,z3) #[25, 18, 14, 4, 11]

#############################################################################################################################
#存活个体数
w = 50
live = []
Zx_w = [19000 * 0.7, 26000 * 0.7, 38000 * 0.7, 6000 * 0.7, 2000 * 0.7]
Zx_num = [25, 18, 6, 4, 11]

#初始化适应个体
while len(live) != w:
# 总民用船的数目
random_W = np.array([random.random(),random.random(),random.random(),random.random(),random.random()])
# 初始个体数目
num_Zx = np.trunc(random_W*Zx_num).astype(np.int8)#取整

# # 总民用船船只
sum_Z_ship = sum_ship_list(Zx_w, num_Zx)
# print("总民用船船只\n" + str(sum_Z_ship) + '\n')
dict_ship_Z = ship_list2dict_z(sum_Z_ship,num_Zx[0],num_Zx[1],num_Zx[2],num_Zx[3],num_Zx[4])# 船变字典
dict_ship_Z['Z3_'+str(num_Zx[2])] = 2494.836677158
sum_Z_ship_dict = dict(kv_sort(dict_ship_Z))# 排序
# print("民用船船只字典和个数\n" + str(sum_Z_ship_dict) + '\n')
# print("剩余的包裹\n" + str(mod_dist) + '\n')
# print(mod_dist)
moddd = kv_deng(mod_dist)

shiying_ship, is_sure = zhuangzai_xuanze_ship(moddd, sum_Z_ship_dict)
# print("是否适应\n" + str(is_sure) + '\n')
moddd = kv_deng(mod_dist)

if is_sure:
num_Zx[2] += 1
live.append(num_Zx.tolist())
# print("装完物体后民用船适应度\n" + str(shiying_ship) + '\n')

# break
# print("适应个体:"+str(live)+'\n')

#############################################################################################################################
# 计算适应度
t0 = surt_time(live[0])
print("计算适应度::"+str(t0))
# 测试交配
# print(live[0],live[1])
# print(jiaopei_1(live[0],live[1]))
# print(jiaopei_2(live[0],live[1]))
#
# #测试变异
# print(live[0])
# print(bianyi(live[0],0.8))
#
# 测试是否满足适应
# print(is_surt_manzai(live[4],mod_dist))
#
#############################################################################################################################
# 遗传的实现
 min_t = []
 duiying_num = []
 for i in range(100):
     the_all_son = live
     bianyilv = 0.2
     for i in range(w):
         #变异
         bianyi_son,is_bianyi = bianyi(live[i],bianyilv)
         if is_bianyi:
            the_all_son.append(bianyi_son)
    # print(len(the_all_son))

     #交配
     random.shuffle(live)
     for i in range(0,w-1,2):
         son = jiaopei_1(live[i],live[i+1])
         the_all_son.append(son)
         son1,son2 = jiaopei_2(live[i],live[i+1])
         the_all_son.append(son1)
         the_all_son.append(son2)
         # the_all_son.append()
    # print(len(the_all_son))


     t0_all = []
     surt_son = []
     for i in range(len(the_all_son)):
         if is_surt_manzai(the_all_son[i],mod_dist):
             t0_all.append(surt_time(the_all_son[i]))
             surt_son.append(the_all_son[i])
     #改进遗传算法种的实现
     # chongfu = []
     # for i in range(len(surt_son)):
     #     for j in range(i,len(surt_son)):
     #         if i!=j and surt_son[i]==surt_son[j]:
     #             # print(j)
     #             chongfu.append(i)
     # # print(t0_all)
     # # print(surt_son)
     #
     # for i in range(len(chongfu),0,-1):
     #     surt_son.pop(i)
    #     t0_all.pop(i)

     # print("变异后个体:"+str(len(the_all_son)))
     num_ship = []
     index = []
     for i in range(len(surt_son)):
         # if t0_all[i] == 21:
         #     print(surt_son[i])
         num_ship.append(sum(surt_son[i]))
         index.append(i)
     # print(t0_all)
     # print(num_ship)

     x = np.vstack((np.array(t0_all),np.array(num_ship),np.array(index)))
     # 第一行排序
     x = x.T[np.lexsort(x[::-1,:])].T
     min_t.append(x[0][0])
     duiying_num.append(x[1][0])

     youxiu_geti = []
     for i in range(w):
         youxiu_geti.append(surt_son[x[2][i]])
     print(youxiu_geti)
     print("时间:"+str(x[0]))
     print("船数:"+str(x[1]))
     live = youxiu_geti

plt.plot(duiying_num, label='种群中对应船只数量', )

plt.plot(min_t, label='种群中最小装载时间', color="r")
plt.xlabel("迭代次数", fontproperties=my_font)
plt.title("适应度变化趋势", fontproperties=my_font)
plt.legend(prop=my_font, loc=0)
plt.grid(alpha=0.3, linestyle="--")  # alpha为透明度 0-1

plt.show()
#############################################################################################################################
#####遍历Zx_num = [25, 18, 14, 4, 11]
# file = 'testaaaa.txt'
# with open(file, 'a+') as f:
#     for i_a in range(25):
#         for i_b in range(18):
#             for i_c in range(6):
#                 for i_d in range(4):
#                     for i_e in range(11):
#                         # print(i_a,i_b,i_c,i_d,i_e)
#                         if is_surt_manzai([i_a,i_b,i_c,i_d,i_e], mod_dist):
#                             tqqq = surt_time([i_a,i_b,i_c,i_d,i_e])
#                             f.write(str(tqqq)+","+str([i_a,i_b,i_c,i_d,i_e])+","+str(sum([i_a,i_b,i_c,i_d,i_e]))+"\n")
#                             print("*"*20+str(tqqq))
############################################################################################################################


\end{lstlisting}
\end{document}